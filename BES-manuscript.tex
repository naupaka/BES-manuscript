% Options for packages loaded elsewhere
\PassOptionsToPackage{unicode}{hyperref}
\PassOptionsToPackage{hyphens}{url}
\PassOptionsToPackage{dvipsnames,svgnames,x11names}{xcolor}
%
\documentclass[
  letterpaper,
  DIV=11,
  numbers=noendperiod]{scrartcl}

\usepackage{amsmath,amssymb}
\usepackage{iftex}
\ifPDFTeX
  \usepackage[T1]{fontenc}
  \usepackage[utf8]{inputenc}
  \usepackage{textcomp} % provide euro and other symbols
\else % if luatex or xetex
  \usepackage{unicode-math}
  \defaultfontfeatures{Scale=MatchLowercase}
  \defaultfontfeatures[\rmfamily]{Ligatures=TeX,Scale=1}
\fi
\usepackage{lmodern}
\ifPDFTeX\else  
    % xetex/luatex font selection
\fi
% Use upquote if available, for straight quotes in verbatim environments
\IfFileExists{upquote.sty}{\usepackage{upquote}}{}
\IfFileExists{microtype.sty}{% use microtype if available
  \usepackage[]{microtype}
  \UseMicrotypeSet[protrusion]{basicmath} % disable protrusion for tt fonts
}{}
\makeatletter
\@ifundefined{KOMAClassName}{% if non-KOMA class
  \IfFileExists{parskip.sty}{%
    \usepackage{parskip}
  }{% else
    \setlength{\parindent}{0pt}
    \setlength{\parskip}{6pt plus 2pt minus 1pt}}
}{% if KOMA class
  \KOMAoptions{parskip=half}}
\makeatother
\usepackage{xcolor}
\setlength{\emergencystretch}{3em} % prevent overfull lines
\setcounter{secnumdepth}{5}
% Make \paragraph and \subparagraph free-standing
\makeatletter
\ifx\paragraph\undefined\else
  \let\oldparagraph\paragraph
  \renewcommand{\paragraph}{
    \@ifstar
      \xxxParagraphStar
      \xxxParagraphNoStar
  }
  \newcommand{\xxxParagraphStar}[1]{\oldparagraph*{#1}\mbox{}}
  \newcommand{\xxxParagraphNoStar}[1]{\oldparagraph{#1}\mbox{}}
\fi
\ifx\subparagraph\undefined\else
  \let\oldsubparagraph\subparagraph
  \renewcommand{\subparagraph}{
    \@ifstar
      \xxxSubParagraphStar
      \xxxSubParagraphNoStar
  }
  \newcommand{\xxxSubParagraphStar}[1]{\oldsubparagraph*{#1}\mbox{}}
  \newcommand{\xxxSubParagraphNoStar}[1]{\oldsubparagraph{#1}\mbox{}}
\fi
\makeatother


\providecommand{\tightlist}{%
  \setlength{\itemsep}{0pt}\setlength{\parskip}{0pt}}\usepackage{longtable,booktabs,array}
\usepackage{calc} % for calculating minipage widths
% Correct order of tables after \paragraph or \subparagraph
\usepackage{etoolbox}
\makeatletter
\patchcmd\longtable{\par}{\if@noskipsec\mbox{}\fi\par}{}{}
\makeatother
% Allow footnotes in longtable head/foot
\IfFileExists{footnotehyper.sty}{\usepackage{footnotehyper}}{\usepackage{footnote}}
\makesavenoteenv{longtable}
\usepackage{graphicx}
\makeatletter
\def\maxwidth{\ifdim\Gin@nat@width>\linewidth\linewidth\else\Gin@nat@width\fi}
\def\maxheight{\ifdim\Gin@nat@height>\textheight\textheight\else\Gin@nat@height\fi}
\makeatother
% Scale images if necessary, so that they will not overflow the page
% margins by default, and it is still possible to overwrite the defaults
% using explicit options in \includegraphics[width, height, ...]{}
\setkeys{Gin}{width=\maxwidth,height=\maxheight,keepaspectratio}
% Set default figure placement to htbp
\makeatletter
\def\fps@figure{htbp}
\makeatother
% definitions for citeproc citations
\NewDocumentCommand\citeproctext{}{}
\NewDocumentCommand\citeproc{mm}{%
  \begingroup\def\citeproctext{#2}\cite{#1}\endgroup}
\makeatletter
 % allow citations to break across lines
 \let\@cite@ofmt\@firstofone
 % avoid brackets around text for \cite:
 \def\@biblabel#1{}
 \def\@cite#1#2{{#1\if@tempswa , #2\fi}}
\makeatother
\newlength{\cslhangindent}
\setlength{\cslhangindent}{1.5em}
\newlength{\csllabelwidth}
\setlength{\csllabelwidth}{3em}
\newenvironment{CSLReferences}[2] % #1 hanging-indent, #2 entry-spacing
 {\begin{list}{}{%
  \setlength{\itemindent}{0pt}
  \setlength{\leftmargin}{0pt}
  \setlength{\parsep}{0pt}
  % turn on hanging indent if param 1 is 1
  \ifodd #1
   \setlength{\leftmargin}{\cslhangindent}
   \setlength{\itemindent}{-1\cslhangindent}
  \fi
  % set entry spacing
  \setlength{\itemsep}{#2\baselineskip}}}
 {\end{list}}
\usepackage{calc}
\newcommand{\CSLBlock}[1]{\hfill\break\parbox[t]{\linewidth}{\strut\ignorespaces#1\strut}}
\newcommand{\CSLLeftMargin}[1]{\parbox[t]{\csllabelwidth}{\strut#1\strut}}
\newcommand{\CSLRightInline}[1]{\parbox[t]{\linewidth - \csllabelwidth}{\strut#1\strut}}
\newcommand{\CSLIndent}[1]{\hspace{\cslhangindent}#1}

\usepackage{booktabs}
\usepackage{longtable}
\usepackage{array}
\usepackage{multirow}
\usepackage{wrapfig}
\usepackage{float}
\usepackage{colortbl}
\usepackage{pdflscape}
\usepackage{tabu}
\usepackage{threeparttable}
\usepackage{threeparttablex}
\usepackage[normalem]{ulem}
\usepackage{makecell}
\usepackage{xcolor}
\usepackage{lineno,setspace}
\linenumbers
\doublespacing
\KOMAoption{captions}{tableheading}
\makeatletter
\@ifpackageloaded{caption}{}{\usepackage{caption}}
\AtBeginDocument{%
\ifdefined\contentsname
  \renewcommand*\contentsname{Table of contents}
\else
  \newcommand\contentsname{Table of contents}
\fi
\ifdefined\listfigurename
  \renewcommand*\listfigurename{List of Figures}
\else
  \newcommand\listfigurename{List of Figures}
\fi
\ifdefined\listtablename
  \renewcommand*\listtablename{List of Tables}
\else
  \newcommand\listtablename{List of Tables}
\fi
\ifdefined\figurename
  \renewcommand*\figurename{Figure}
\else
  \newcommand\figurename{Figure}
\fi
\ifdefined\tablename
  \renewcommand*\tablename{Table}
\else
  \newcommand\tablename{Table}
\fi
}
\@ifpackageloaded{float}{}{\usepackage{float}}
\floatstyle{ruled}
\@ifundefined{c@chapter}{\newfloat{codelisting}{h}{lop}}{\newfloat{codelisting}{h}{lop}[chapter]}
\floatname{codelisting}{Listing}
\newcommand*\listoflistings{\listof{codelisting}{List of Listings}}
\makeatother
\makeatletter
\makeatother
\makeatletter
\@ifpackageloaded{caption}{}{\usepackage{caption}}
\@ifpackageloaded{subcaption}{}{\usepackage{subcaption}}
\makeatother

\ifLuaTeX
  \usepackage{selnolig}  % disable illegal ligatures
\fi
\usepackage{bookmark}

\IfFileExists{xurl.sty}{\usepackage{xurl}}{} % add URL line breaks if available
\urlstyle{same} % disable monospaced font for URLs
\hypersetup{
  pdftitle={A direct comparison between field-measured and sensor-based estimates of soil carbon dioxide flux across six National Ecological Observatory Network sites enabled by the neonSoilFlux R package},
  pdfauthor={, , , , , , , , , , , and },
  colorlinks=true,
  linkcolor={blue},
  filecolor={Maroon},
  citecolor={Blue},
  urlcolor={Blue},
  pdfcreator={LaTeX via pandoc}}


\title{A direct comparison between field-measured and sensor-based
estimates of soil carbon dioxide flux across six National Ecological
Observatory Network sites enabled by the \texttt{neonSoilFlux} R
package}
\author{John Zobitz\textsuperscript{1} \and Ed
Ayres\textsuperscript{2} \and Zoey Werbin\textsuperscript{3} \and Ridwan
Abdi\textsuperscript{1} \and Natalie
Ashburner-Wright\textsuperscript{4} \and Lillian
Brown\textsuperscript{4} \and Ryan
Frink-Sobierajski\textsuperscript{4} \and Lajntxiag
Lee\textsuperscript{1} \and Dijonë
Mehmeti\textsuperscript{1} \and Christina
Tran\textsuperscript{4} \and Ly Xiong\textsuperscript{1} \and Naupaka
Zimmerman\textsuperscript{4}}
\date{}

\begin{document}
\maketitle


\textsuperscript{1} Augsburg University, 2211 Riverside Avenue,
Minneapolis, MN 55454\\
\textsuperscript{2} National Ecological Observatory Network, 1685 38th
Street, Suite 100, Boulder, CO 80301\\
\textsuperscript{3} Boston University, 5 Cummington Street, Boston, MA
02215\\
\textsuperscript{4} University of San Francisco, 2130 Fulton Street, San
Francisco, CA 94117

\section*{Acknowledgments}\label{acknowledgments}
\addcontentsline{toc}{section}{Acknowledgments}

JZ acknowledges Kathleen O'Rourke for code development. NZ thanks
technical staff at USF for support with field gear assembly and
shipping. We thank the NEON field staff and assignable assets teams for
facilitating each of the six NEON site visits. We are grateful to LI-COR
technical staff for helpful discussions about optimal sampling methods.
This work was supported by NSF DEB grant \#2017829 awarded to JZ, and
NSF DEB grant \#2017860 awarded to NZ.

\section*{Conflict of Interest
Statements}\label{conflict-of-interest-statements}
\addcontentsline{toc}{section}{Conflict of Interest Statements}

None of the authors have a financial, personal, or professional conflict
of interest related to this work.

\section*{Author Contributions}\label{author-contributions}
\addcontentsline{toc}{section}{Author Contributions}

Conceptualization: JZ, NZ; Methodology: EA, JZ, NZ; Software: JZ, NZ,
ZW, E A, DM, RA, LX, LL; Validation: JZ, NZ; Formal Analysis: JZ, NZ,
DM, RA, LX, LL; Investigation: JZ, NZ, RF-S, CT, NA-W, LB; Resources:
JZ, NZ; Data curation: JZ, NZ, DM, LX; Writing -- original draft: JZ,
NZ; Writing -- review and editing: JZ, NZ, ZW, EA, CT, DM, LX,;
Visualization: JZ, NZ, DM, RA, LX; Supervision: JZ; NZ; Project
Administration: JZ; NZ; Funding Acquisition: JZ; NZ

\section*{Data Availability}\label{data-availability}
\addcontentsline{toc}{section}{Data Availability}

Data available from the Zenodo LINK
http://dx.doi.org/10.5061/dryad.41qh7 (Kiere \& Drummond 2016).''

\newpage

\section{Abstract}\label{abstract}

A key component of constraining the uncertainty of the terrestrial
carbon sink is quantification of terrestrial soil carbon fluxes, which
vary across time and ecosystem type. One method for the estimation of
these fluxes and their associated uncertainties is the flux gradient
method, which can be calculated via a variety of existing approaches.
Robust estimation of soil carbon fluxes on a sub-daily level requires
measurements of soil CO\(_{2}\) concentration, water content,
temperature, and other environmental measurements and soil properties.
These data are publicly available from the National Ecological
Observatory Network at sites spanning a range of 20 different
ecoclimatic domains across the continental United States, Puerto Rico,
Alaska, and Hawai'i. We present an R software package
(\texttt{neonSoilFlux}) that acquires NEON soil environmental data and
computes soil carbon flux at a half-hourly time step at a user-specified
NEON site and month in a tidy data format. To validate the computed
fluxes, we visited six focal NEON sites and measured soil carbon fluxes
using a closed-dynamic chamber approach. The validation confirmed that a
primary challenge in reducing soil carbon flux uncertainty is correctly
characterizing diffusivity and soil water content across the soil
profile. Outputs from the \texttt{neonSoilFlux} package contribute to
existing databases of soil carbon flux measurements, providing near
real-time estimates of a critical component of the terrestrial carbon
cycle.

\subsection{Keywords}\label{keywords}

Soil carbon, carbon dioxide, flux gradient, carbon cycle, field
validation, soil respiration, ecosystem variability, diffusion

\section{Data for peer review}\label{data-for-peer-review}

Anonymous data and code for peer review is available here: LINK

\section{Introduction}\label{introduction}

Soils contain the largest reservoir of terrestrial carbon (Jobbágy \&
Jackson, 2000). A critical component of this reservoir is soil organic
matter, the accumulation of which is influenced by biotic factors such
as above-ground plant inputs (Jackson et al., 2017). These inputs in
turn are influenced by environmental factors such as growing season
length, temperature, and moisture (Desai et al., 2022), which also
affect the breakdown of soil organic matter and its return to the
atmosphere. Across heterogeneous terrestrial landscapes, the interplay
between these biotic and abiotic factors influence the size of the soil
contribution to the terrestrial carbon sink (Friedlingstein et al.,
2023). However, the heterogeneity of these processes across diverse
ecosystems in the context of rapid environmental change leads to large
uncertainty in the magnitude of this sink in the future, and thus a
pressing need to quantify changes in soil carbon pools and fluxes across
scales.

Ecological observation networks such as the United States' National
Ecological Observatory Network (NEON) and others (e.g.~FLUXNET or the
Integrated Carbon Observation System) present a significant advancement
in the nearly continuous observation of biogeochemical processes at the
continental scale. Notably, at 47 terrestrial sites across the
continental United States, NEON provides half-hourly measurements of
soil CO\(_{2}\) concentration, temperature, and moisture at different
vertical depths. Each of these NEON sites also encompasses measurements
of the cumulative sum of all ecosystem carbon fluxes in an airshed using
the eddy covariance technique (Baldocchi, 2014). Soil observations
provided by NEON are on the same timescale and standardized with eddy
covariance measurements from FLUXNET. These types of nearly continuous
observational data (NEON and FLUXNET) can be used to reconcile
differences between model-derived or data-estimated components of
ecosystem carbon flux (Jian et al., 2022; Luo et al., 2011; Phillips et
al., 2017; J. Shao et al., 2015; P. Shao et al., 2013; Sihi et al.,
2016).

Estimated or observed soil carbon fluxes are a key metric for
understanding change in soil carbon pools over time (Bond-Lamberty et
al., 2024). A soil carbon flux to the atmosphere (\(F_{S}\), units
\(\mu\)mol m\(^{-2}\) s\(^{-1}\)), represents the aggregate process of
transfer of soil CO\(_{2}\) to the atmosphere from physical and
biological processes (e.g.~diffusion and respiration). Soil carbon
fluxes can be assumed to encompass soil carbon respiration from
autotrophic or heterotrophic sources (Davidson et al., 2006), typically
assumed to be static across the soil biome and modeled with a
exponential \(Q_{10}\) paradigm (Bond-Lamberty et al., 2004; Chen \&
Tian, 2005; Hamdi et al., 2013).

One method by which \(F_{S}\) is measured in the field is through the
use of soil chambers in a closed, well-mixed system (Norman et al.,
1997) with headspace trace gas concentrations measured with an infrared
gas analyzer (IRGA). \(F_{S}\) can also be estimated from soil
CO\(_{2}\) measurements at different depths in the soil using the
flux-gradient method (Maier \& Schack-Kirchner, 2014). This method is an
approach that uses conservation of mass to calculate flux at a vertical
soil depth \(z\) at steady state by applying Fick's law of diffusion. A
simplifying assumption for the flux-gradient method is that there is no
mass transfer in the other spatial dimensions \(x\) and \(y\) (Maier \&
Schack-Kirchner, 2014). The diffusivity profile, a key component of this
calculation, varies across the soil depth as a function of soil
temperature, soil volumetric water content, atmospheric air pressure,
and soil bulk density (Millington \& Shearer, 1971; Moldrup et al.,
1999; Sallam et al., 1984).

Databases such as the Soil Respiration Database (SRDB) or the Continuous
Soil Respiration Database (COSORE) add to the growing network of
resources for making collected observations of soil fluxes available to
other workers (Bond-Lamberty, 2018; Bond-Lamberty et al., 2020;
Bond-Lamberty \& Thomson, 2010; Jian et al., 2021; Jiang et al., 2024).
However, these databases currently encompass primarily direct soil
measurements of fluxes (i.e.~those using methods like the closed-chamber
method described above). Currently, NEON provides all measurements to
calculate \(F_{S}\) from Fick's law, but soil flux as a derived data
product was descoped from the initial network launch due to budget
constraints (Berenbaum et al., 2015). Deriving estimates of \(F_{S}\)
using continuous sensor data across NEON sites thus represents a high
priority.

This study describes an R software package, \texttt{neonSoilFlux}, that
can be used to derive a standardized estimate of \(F_{S}\) at all
terrestrial NEON sites. After calculating these flux estimates, we then
validated them against direct chamber-based field observations of soil
carbon dioxide flux from a subset of terrestrial NEON sites spanning six
states.

Key objectives of this study are to:

\begin{enumerate}
\def\labelenumi{\arabic{enumi}.}
\tightlist
\item
  Apply the flux-gradient method to estimate soil CO\(_{2}\) flux from
  continuous sensor measurements across NEON sites.
\item
  Benchmark estimated soil carbon fluxes against field measurements
  (e.g.~direct chamber measurements of soil flux).
\item
  Identify sources of error in the flux-gradient approach across diverse
  sites in order to guide future work.
\end{enumerate}

\section{Materials and Methods}\label{materials-and-methods}

\subsection{Field methods}\label{field-methods}

\subsubsection{Focal NEON Sites}\label{focal-neon-sites}

In order to acquire field data to validate model predictions of flux, we
selected six terrestrial NEON sites for analysis. We conducted field
measurement campaigns at these sites, which span a range of
environmental gradients and terrestrial domains
(Table~\ref{tbl-neon-sites}). SJER, SRER, and WREF were visited during
May and June of 2022, and WOOD, KONZ, and UNDE during May and June of
2024.

Over the course of two field campaigns in 2022 and 2024, we conducted
week-long visits at each site. In consultation with NEON field staff, we
first selected a specific plot in the soil sampling array to maximize
the concurrent availability of sensor data.

\subsubsection{Soil collar placement}\label{soil-collar-placement}

Either one (2022 sampling campaign) or two (2024 sampling campaign) PVC
soil collars (20.1 cm inside diameter) were installed in close proximity
to the permanent NEON soil sensors at each site
(Figure~\ref{fig-collar-images}). The soil plot where measurements were
taken was chosen at each site in consultation with NEON staff to
maximize likelihood of quality soil sensor measurements during the
duration of the IRGA measurements at each site. After installation,
collar(s) were left to equilibrate for approximately 24 hours prior to
measurements being taken.

\subsubsection{\texorpdfstring{Infrared gas analyzer measurements of
soil CO\(_{2}\)
flux}{Infrared gas analyzer measurements of soil CO\_\{2\} flux}}\label{infrared-gas-analyzer-measurements-of-soil-co_2-flux}

In 2022, we then made measurements of flux on an hourly interval for 8
hours each day. Measurements were taken from roughly 8 am to 4 pm, with
the time interval selected to capture the majority of the diurnal
gradient of soil temperature each day. These measurements were made
using a LI-6800 infrared gas analyzer instrument (LI-COR Environmental,
Lincoln, NE) fitted with a soil chamber attachment (attachment 6800-09).
In 2024, we again used the same LI-6800 instrument, but made half-hourly
measurements over an approximately 8 hour period. In addition, we also
installed a second collar and used a second instrument, an LI-870
CO\(_{2}\) IRGA, connected to an automated robotic chamber (LI-COR
chamber 8200-104) controlled by an LI-8250 multiplexer, to make
automated measurements. The multiplexer was configured to take
half-hourly measurements 24 hours a day for the duration of our sampling
bout at each site. Each instrument was paired with a soil temperature
and moisture probe (Stevens HydraProbe, Stevens Water, Portland, OR)
that was used to make soil temperature and moisture measurements
concurrent with the CO\(_{2}\) flux measurements. Chamber volumes were
set by measuring collar offsets at each site. System checks were
conducted daily for the LI-6800 and weekly for the LI-8250. Instruments
were factory calibrated before each field season.

\begin{figure}

\centering{

\includegraphics{figures/collar-images.jpeg}

}

\caption{\label{fig-collar-images}Spatial layout of field sampling using
a closed-dynamic chamber setup at a representative NEON site (KONZ).
Left image shows collars (white arrows) and permanent soil sensor
installation (black arrow) and right image shows the LI-6800
(foreground) and LI-8200-104 (background) instruments placed on the
collars.}

\end{figure}%

\scriptsize

\begin{longtable}[]{@{}
  >{\raggedright\arraybackslash}p{(\columnwidth - 16\tabcolsep) * \real{0.1111}}
  >{\raggedright\arraybackslash}p{(\columnwidth - 16\tabcolsep) * \real{0.1111}}
  >{\raggedright\arraybackslash}p{(\columnwidth - 16\tabcolsep) * \real{0.1111}}
  >{\raggedright\arraybackslash}p{(\columnwidth - 16\tabcolsep) * \real{0.1111}}
  >{\raggedright\arraybackslash}p{(\columnwidth - 16\tabcolsep) * \real{0.1111}}
  >{\raggedright\arraybackslash}p{(\columnwidth - 16\tabcolsep) * \real{0.1111}}
  >{\raggedright\arraybackslash}p{(\columnwidth - 16\tabcolsep) * \real{0.1111}}
  >{\raggedright\arraybackslash}p{(\columnwidth - 16\tabcolsep) * \real{0.1111}}
  >{\raggedright\arraybackslash}p{(\columnwidth - 16\tabcolsep) * \real{0.1111}}@{}}

\caption{\label{tbl-neon-sites}Listing of NEON sites studied for field
work and analysis. \(\overline{T_{S}}\): average soil temperature during
field measurements. \(\overline{SWC}\): average soil water content
during field measurements. Soil plot refers to the particular location
in the soil sensor array (denoted as HOR by NEON) where field
measurements were made.}

\tabularnewline

\toprule\noalign{}
\begin{minipage}[b]{\linewidth}\raggedright
Site (NEON site ID)
\end{minipage} & \begin{minipage}[b]{\linewidth}\raggedright
Location
\end{minipage} & \begin{minipage}[b]{\linewidth}\raggedright
Ecosystem type
\end{minipage} & \begin{minipage}[b]{\linewidth}\raggedright
Mean annual temperature
\end{minipage} & \begin{minipage}[b]{\linewidth}\raggedright
\(\overline{T_{S}}\) (\(^{\circ}\))
\end{minipage} & \begin{minipage}[b]{\linewidth}\raggedright
Mean annual precipitation
\end{minipage} & \begin{minipage}[b]{\linewidth}\raggedright
\(\overline{SWC}\) (\%)
\end{minipage} & \begin{minipage}[b]{\linewidth}\raggedright
Field measurement dates
\end{minipage} & \begin{minipage}[b]{\linewidth}\raggedright
Soil plot
\end{minipage} \\
\midrule\noalign{}
\endhead
\bottomrule\noalign{}
\endlastfoot
Santa Rita Experimental Range (SRER) & 31.91068, -110.83549 & Shrubland
& 19.3\(^{\circ}\)C & 47.6\(^{\circ}\) & 346 mm & 4.0\% & 29 May 2024 -
01 June 2024 & 004 \\
San Joaquin Experimental Range (SJER) & 37.10878, -119.73228 & Oak
woodland & 16.4\(^{\circ}\)C & 41.7\(^{\circ}\) & 540 mm & 1.2\% & 01
June 2022 - 04 June 2022 & 005 \\
Wind River Experimental Forest (WREF) & 45.82049, -121.95191 & Evergreen
forest & 9.2\(^{\circ}\)C & 15.3\(^{\circ}\) & 2225 mm & 27.2\% & 07
June 2022 - 09 June 2022 & 001 \\
Chase Lake National Wildlife Refuge (WOOD) & 47.1282, -99.241334 &
Restored prairie grassland & 4.9\(^{\circ}\)C & 14.9\(^{\circ}\) & 495
mm & 14.9\% & 03 June 2024 - 09 June 2024 & 001 \\
Konza Prairie Biological Station (KONZ) & 39.100774, -96.563075 &
Tallgrass Prairie & 12.4\(^{\circ}\)C & 23.4\(^{\circ}\) & 870 mm &
23.4\% & 29 May 2024 - 01 June 2024 & 001 \\
University of Notre Dame Environmental Research Center (UNDE) &
46.23391, -89.537254 & Deciduous forest & 4.3\(^{\circ}\) &
13.0\(^{\circ}\) & 802 mm & 13.0\% & 22 May 2024 - 25 May 2024 & 004 \\

\end{longtable}

\normalsize

\subsubsection{Post-collection processing of field
data}\label{post-collection-processing-of-field-data}

We used LI-COR SoilFluxPro software (v 5.3.1) to assess the data after
collection and to inform sampling parameters. We checked appropriateness
of dead band and measurement durations using built-in evaluation tools.
Based on this, the deadband period was set for 30-40 seconds, depending
on the site, and the measurement duration was 180 seconds with a 30
second pre-purge and a 30 second post-purge at most sites, and a 90 sec
pre- and post-purge at sites with higher humidity due to recent
precipitation events. We also assessed the R\(^{2}\) of linear and
exponential model fits to measured CO\(_{2}\) to verify measurement
quality.

\subsection{\texorpdfstring{\texttt{neonSoilFlux} R
package}{neonSoilFlux R package}}\label{neonsoilflux-r-package}

We developed an R package (\texttt{neonSoilFlux};
https://CRAN.R-project.org/package=neonSoilFlux) to compute half-hourly
soil carbon fluxes and uncertainties from NEON data. The objective of
the \texttt{neonSoilFlux} package is a unified workflow
(Figure~\ref{fig-package-diagram}) for soil data acquisition and
analysis that supplements the existing data acquisition R package
\texttt{neonUtilities} (LINK TO BE ADDED AFTER PEER REVIEW).

\begin{figure}

\centering{

\includegraphics{figures/neonSoilFluxOutline.png}

}

\caption{\label{fig-package-diagram}Diagram of \texttt{neonSoilFlux} R
package. For a given month, year, and NEON site (orange parallelogram),
the package acquires all relevant data to compute \(F_{S}\) using the
\texttt{neonUtilities} R package (green rectangle). Data are gap-filled
according to reported QA flags and interpolated to the same measurement
depth before computing the soil flux, uncertainties, and final QA flags
(blue rectangles). The package reports the associated soil flux,
uncertainties, and quality assurance (QA) flags for the user (purple
parallelogram).}

\end{figure}%

At a given NEON observation there are five replicate soil plots, each
with measurements of soil CO\(_{2}\) concentration, soil temperature,
and soil moisture at different depths (Figure~\ref{fig-model-diagram}).
The \texttt{neonSoilFlux} package acquires measured soil water content
(National Ecological Observatory Network (NEON), 2024e), soil CO\(_{2}\)
concentration (National Ecological Observatory Network (NEON), 2024b),
barometric pressure from the nearby tower (National Ecological
Observatory Network (NEON), 2024a), soil temperature (National
Ecological Observatory Network (NEON), 2024d), and soil properties
(e.g.~bulk density) (National Ecological Observatory Network (NEON),
2024c). The static soil properties were collected from a nearby soil pit
during site characterization and are assumed to be constant at each
site.

\begin{figure}

\centering{

\includegraphics{figures/model-diagram.pdf}

}

\caption{\label{fig-model-diagram}Model diagram for data workflow for
the \texttt{neonSoilFlux} R package. a) Acquire: Data are obtained from
given NEON location and horizontal sensor location, which includes soil
water content, soil temperature, CO\(_{2}\) concentration, and
atmospheric pressure. All data are screened for quality assurance; if
gap-filling of missing data occurs, it is flagged for the user. b) Any
belowground data are then harmonized to the same depth as CO\(_{2}\)
concentrations using linear regression. c) The flux across a given depth
is computed via Fick's law, denoted with \(F_{ijk}\), where \(i\),
\(j\), or \(k\) are either 0 or 1 denoting the layers the flux is
computed across (\(i\) = closest to surface, \(k\) = deepest).
\(F_{000}\) represents a flux estimate where the gradient \(dC/dz\) is
the slope of a linear regression of CO\(_{2}\) with depth.}

\end{figure}%

The workflow to compute a value of \(F_{S}\) with \texttt{neonSoilFlux}
consists of three primary steps, illustrated in
Figure~\ref{fig-model-diagram}. First, NEON data are acquired for a
given site and month via the \texttt{neonUtilities} R package (yellow
parallelogram and green rectangle in Figure~\ref{fig-package-diagram}
and Panel a in Figure~\ref{fig-model-diagram}). Acquired environmental
data can be exported to a comma separated value file for additional
analysis. Quality assurance (QA) flags are reported as an indicator
variable.

The second step is harmonizing the data to compute soil fluxes across
soil layers. This step consists of three different actions (blue
rectangles in Figure~\ref{fig-package-diagram} and Panel b in
Figure~\ref{fig-model-diagram}). If a given observation by NEON is
reported as not passing a quality assurance check, we applied a gap
filling method to replace that measurement with its monthly mean at that
same depth (Section~\ref{sec-gapfilling}). Belowground measurements of
soil water and soil temperature are then interpolated to the same depth
as soil CO\(_{2}\) measurements. The diffusivity
(Section~\ref{sec-compute-diffusivity}) and soil flux across different
soil layers (Section~\ref{sec-compute-soil-flux}) are then computed.

The third and final step is computing a surface soil flux through
extrapolation to the surface (purple parallelogram in
Figure~\ref{fig-package-diagram} and Panel c in
Figure~\ref{fig-model-diagram}). Uncertainty on a soil flux measurement
is computed through quadrature. An aggregate quality assurance (QA) flag
for each environmental measurement is also reported, representing if any
gap-filled measurements were used in the computation of a soil flux.
Within the soil flux-gradient method, several different approaches can
be used to derive a surface flux (Maier \& Schack-Kirchner, 2014); the
\texttt{neonSoilFlux} package reports four different possible values for
soil surface flux (Section~\ref{sec-compute-soil-flux}).

\subsubsection{Gap-filling routine}\label{sec-gapfilling}

NEON reports QA flags as a binary value for a given measurement and
half-hourly time interval. We replaced any flagged measurements at a
location's spatial depth \(z\) with a bootstrapped sample of the monthly
mean for all un-flagged measurements for that month. These measurements
are represented by the vector \(\mathbf{m}\), standard errors
\(\boldsymbol\sigma\), and the 95\% confidence interval (the so-called
expanded uncertainty, Farrance \& Frenkel (2012))
\(\boldsymbol\epsilon\). All of these vectors have length \(M\). We have
that \(\vec{\sigma}_{i}\leq\vec{\epsilon}_{i}\). We define the bias as
\(\mathbf{b}=\sqrt{\boldsymbol\epsilon^{2}-\boldsymbol\sigma^{2}}\).

We generate a vector of bootstrap samples of the distribution of the
monthly mean \(\overline{\boldsymbol{m}}\) and monthly standard error
\(\overline{\boldsymbol\sigma}\) the following ways:

\begin{enumerate}
\def\labelenumi{\arabic{enumi}.}
\tightlist
\item
  Randomly sample from the uncertainty and bias independently:
  \(\boldsymbol\sigma_{j}\) and the bias \(\mathbf{b}_{k}\) (not
  necessarily the same sample).
\item
  Generate a vector \(\mathbf{n}\) of length \(N\), where
  \(\mathbf{n}_{i}\) is a random sample from a normal distribution with
  mean \(\boldsymbol{m}_{i}\) and standard deviation
  \(\boldsymbol\sigma_{j}\). Since \(M<N\), values from \(\mathbf{m}\)
  will be reused.
\item
  With these \(N\) random samples,
  \(\overline{y}_{i}=\overline{\vec{x}}+\vec{b}_{k}\) and \(s_{i}\) is
  the sample standard deviation of \(\vec{x}\). We expect that
  \(s_{i} \approx \vec{\sigma}_{j}\).
\item
  The reported monthly mean and standard deviation are then computed
  \(\overline{\overline{y}}\) and \(\overline{s}\). Measurements and
  uncertainties that did not pass the QA check are then substituted with
  \(\overline{\overline{y}}\) and \(\overline{s}\).
\end{enumerate}

This gap-filling method described here provides a consistent approach
for each data stream, however we recognize that other gap-filling
alternatives may be warranted for longer-term gaps (e.g.~such as
correlations with other NEON measurement levels and soil plots), or
measurement specific gap-filling routines. We discuss the effect of
gap-filling on our measurements in Section~\ref{sec-discussion}.

\subsubsection{Soil diffusivity}\label{sec-compute-diffusivity}

Soil diffusivity \(D_{a}\) at a given measurement depth is the product
of the diffusivity in free air \(D_{a,0}\) (m\(^{2}\) s\(^{-1}\)) and
the tortuosity \(\xi\) (no units) (Millington \& Shearer, 1971).

We compute \(D_{a,0}\) with Equation \ref{eq:da0}:

\begin{equation}
  D_{a,0} = 0.0000147 \cdot \left( \frac{T_{i} + 273.15}{293.15} \right)^{1.75} \cdot \left( \frac{P}{101.3} \right)
  \label{eq:da0}
\end{equation}

where \(T_{i}\) is soil temperature (\(^\circ\)C) at depth \(i\)
(National Ecological Observatory Network (NEON), 2024d) and \(P\)
surface barometric pressure (kPa) (National Ecological Observatory
Network (NEON), 2024a).

Previous studies by Sallam et al. (1984) and Tang et al. (2003)
demonstrated the sensitivity of modeled \(F_{S}\) depending on the
tortuousity model used to compute diffusivity. At low soil water
content, the choice of tortusoity model may lead to order of magnitude
differences in \(D_{a}\), which in turn affect modeled \(F_{S}\). The
\texttt{neonSoilFlux} package uses two different models for \(\xi\),
representing the extremes reported in Sallam et al. (1984). The first
approach uses the Millington-Quirk model for diffusivity, Equation
\ref{eq:tortuosity-mq} (Millington \& Shearer, 1971):

\begin{equation}
  \xi = \frac{(\phi - SWC_{i})^{10/3}}{\phi^{2}}
  \label{eq:tortuosity-mq}
\end{equation}

In Equation \ref{eq:tortuosity-mq}, \(SWC\) is the soil water content at
depth \(i\) (National Ecological Observatory Network (NEON), 2024e) and
\(\phi\) is the porosity (Equation \ref{eq:porosity}), which in turn is
a function of soil physical properties (National Ecological Observatory
Network (NEON), 2024c):

\begin{equation}
  \phi = \left(1- \frac{\rho_{s}}{\rho_{m}} \right) \left(1-f_{V}\right)
  \label{eq:porosity}
\end{equation}

In Equation \ref{eq:porosity}, \(\rho_{m}\) is the particle density of
mineral soil (2.65 g cm\(^{-3}\)), \(\rho_{s}\) the soil bulk density (g
cm\(^{-3}\)) excluding coarse fragments greater than 2 mm (National
Ecological Observatory Network (NEON), 2024c). The term \(f_{V}\) is a
site-specific value that accounts for the proportion of soil fragments
between 2-20 mm. Soil fragments greater than 20 mm were not estimated
due to limitations in the amount of soil that can be analyzed (National
Ecological Observatory Network (NEON), 2024c). We assume there are no
pores within rocks.

The second approach to calculate \(\xi\) is the Marshall model
(Marshall, 1959), where \(\xi = \phi^{1.5}\), with \(\phi\) defined from
Equation \ref{eq:porosity}.

\subsubsection{Soil flux computation}\label{sec-compute-soil-flux}

We applied Fick's law (Equation \ref{eq:ficks}) to compute the soil flux
\(F_{ij}\) (\(\mu\)mol m\(^{-2}\) s\(^{-1}\)) across two soil depths
\(i\) and \(j\):

\begin{equation}
  F_{ij} = -D_{a} \frac{dC}{dz}
  \label{eq:ficks}
\end{equation}

where \(D_{a}\) is the diffusivity (m\(^{2}\) s\(^{-1}\)) and
\(\frac{dC}{dz}\) is the gradient of CO\(_{2}\) molar concentration
(\(\mu\)mol m\(^{-3}\), so the gradient has units of \(\mu\)mol
m\(^{-3}\) m\(^{-1}\)). The soil surface flux is theoretically defined
by applying Equation \ref{eq:ficks} to measurements collected at the
soil surface and directly below the surface. Measurements of soil
temperature, soil water content, and soil CO\(_{2}\) molar concentration
across the soil profile allow for application of Equation \ref{eq:ficks}
across different soil depths. Each site had three measurement layers, so
we denote the flux between which two layers as a three-digit subscript
\(F_{ijk}\) with indicator variables \(i\), \(j\), and \(k\) indicate if
a given layer was used (written in order of increasing depth), according
to the following:

\begin{itemize}
\tightlist
\item
  \(F_{000}\) is a surface flux estimate using the intercept of the
  linear regression of \(D_{a}\) with depth and the slope from the
  linear regression of CO\(_{2}\) with depth (which represents
  \(\displaystyle \frac{dC}{dz}\) in Fick's Law). Tang et al. (2003)
  used this approach to compute fluxes in an oak-grass savannah.
\item
  \(F_{110}\), \(F_{011}\) are fluxes across the two most shallow layers
  and two deepest layers respectively. The diffusivity used in Fick's
  Law is always at the deeper measurement layer. When used as a surface
  flux estimate we assume CO\(_{2}\) remains constant above this flux
  depth.
\item
  \(F_{101}\) is a surface flux estimate using linear extrapolation
  using concentration measurements between the shallowest and deepest
  measurement layer. Hirano et al. (2003) and Tang et al. (2005) used an
  approach similar to \(F_{101}\) in a temperate deciduous broadleaf
  forest and ponderosa pine forest respectively.
\end{itemize}

Uncertainty in all \(F_{ijk}\) is computed through quadrature (Taylor,
2022).

\subsection{Post processing evaluation}\label{sec-post-process}

Following collection of field measurements and calculation of the soil
fluxes from \texttt{neonSoilFlux} package, we compared measured
\(F_{S}\) based on closed-dynamic chamber measurements with the LI-COR
instruments to a given soil flux calculation from \texttt{neonSoilFlux}
for each site and flux computation method. Statistics included the
associated R\(^{2}\) value, root mean squared error (RMSE), and signal
to noise ratio (SNR), defined as the ratio of a modeled soil flux
(\(F_{ijk}\)) from \texttt{neonSoilFlux} to its quadrature uncertainty
(\(\sigma_{ijk}\)).

We observed that the range of values (e.g.~\(F_{ijk} \pm \sigma_{ijk}\)
was much larger than the measured field flux. We evaluated
\(| F_{S} - F_{ijk} | < (1-\epsilon) \sigma_{ijk}\), where \(F_{S}\) is
a measured field soil flux from the LI-COR 6800 (as the LI-COR 870/8250
was used at only three sites in 2024 but the 6800 was used at all sites
in both years). The parameter \(\epsilon\) was an uncertainty reduction
factor to evaluate how much the quadrature uncertainty could be reduced
while maintaining precision between modeled \(F_{ijk}\) and measured
\(F_{S}\).

Finally, for a half-hourly interval we also computed a \emph{post hoc}
\(D_{a}\) using the LI-COR flux along with the CO\(_{2}\) surface
gradient reported by NEON using the measurement levels closest to the
surface.

\section{Results}\label{results}

Our overall goal was to design and validate an R package to estimate
soil carbon dioxide fluxes fluxes across terrestrial NEON sites using
the flux gradient method. Validation of the approach was based on
comparison of estimated fluxes to field measurements made at six focal
sites. We first present our field measurement results, then the
concordance between the modeled and measured results, and lastly assess
the factors that influenced the success of the modeled approach at a
given site.

\subsection{Field measurements}\label{field-measurements}

We visited six NEON sites in the summers of 2022 and 2024. Using a
closed-dynamic chamber approach, we quantified soil carbon dioxide
fluxes over the course of a week at each site. The sites we visited
ranged substantially in both their annual average temperature and
precipitation as well as their biome type
(Table~\ref{tbl-licor-results}). These differences also influenced the
wide range of observed flux rates across sites. We used a LI-6800 to
take manual hourly measurements at the sites we visited in 2022 (SRER,
SJER, WREF) and half-hourly manual measurements for the sites we visited
in 2024 (UNDE, KONZ, WOOD). In 2024 we also used an automated chamber
system (LI-870/LI-8250) to take half-hourly measurements 24 hours a day,
thereby also capturing nighttime fluxes in addition to the daytime
fluxes also measured with the LI-6800.

\begin{table}

\caption{\label{tbl-licor-results}Summary of measured soil
characteristics and flux results from field measurements across six NEON
sites using a LI-COR 6800 (LI-870/8250 measurements omitted to enable
direct comparability) via the closed-dynamic chamber method. Numeric
values for soil CO\(_{2}\) flux, soil temperature, and volumetric soil
water content (VSWC) are the mean and standard deviation of field
measurements at each site.}

\centering{

\begin{tabular}{ccccc}
\toprule
Site & \makecell[c]{Flux\\ $\mu$mol m$\textsuperscript{-2}$ s$\textsuperscript{-1}$} & \makecell[c]{Soil temp\\ °C} & \makecell[c]{VSWC\\ cm$\textsuperscript{3}$ cm$\textsuperscript{-3}$} & n\\
\midrule
UNDE & 2.55 $\pm$ 0.26 & 14.33 $\pm$ 0.77 & 0.33 $\pm$ 0.02 & 61\\
WOOD & 3.02 $\pm$ 0.4 & 16.01 $\pm$ 1.54 & 0.28 $\pm$ 0.01 & 53\\
WREF & 3.62 $\pm$ 0.3 & 15.34 $\pm$ 1.76 & 0.27 $\pm$ 0.06 & 21\\
KONZ & 6.35 $\pm$ 0.97 & 27.28 $\pm$ 4.14 & 0.37 $\pm$ 0.01 & 44\\
SJER & 0.94 $\pm$ 0.02 & 41.68 $\pm$ 11.22 & 0.01 $\pm$ 0.01 & 32\\
SRER & 0.72 $\pm$ 0.09 & 47.64 $\pm$ 7.46 & 0.04 $\pm$ 0.01 & 32\\
\bottomrule
\end{tabular}

}

\end{table}%

\subsection{\texorpdfstring{Concordance between modelled and measured
soil CO\(_{2}\)
flux}{Concordance between modelled and measured soil CO\_\{2\} flux}}\label{concordance-between-modelled-and-measured-soil-co_2-flux}

The timeseries of the measured fluxes from the LI-COR 6800 and 870/8250
were compared to modeled soil fluxes from the \texttt{neonSoilFlux} R
package (Figure~\ref{fig-flux-results}). We also assessed year-long
estimated flux time series and compared those to field measurements made
at each site (Figure~\ref{fig-flux-results-year}). Results are reported
in local time. Where applicable, sites are displayed from left to right
by increasing soil temperature (Table~\ref{tbl-neon-sites}). Positive
values of the flux indicate that there is a flux moving towards the
surface. Overall, with the exception of SRER (discussed later) the
computed fluxes determined using a variety of plausible methods spanned
the field-measured fluxes, but the specific flux-gradient method that
best approximated field measurements varied by site.

\begin{figure}

\centering{

\includegraphics{figures/flux-results.png}

}

\caption{\label{fig-flux-results}Timeseries of both measured \(F_{S}\)
(yellow circles) and modeled soil fluxes (green or purple lines) by the
\texttt{neonSoilFlux} R package. Fluxes from the \texttt{neonSoilFlux} R
package are separated by the diffusivity model used (Millington-Quirk or
Marshall, Section~\ref{sec-compute-diffusivity}). Vertical axis labels
in the first column represent the measurement levels where the
flux-gradient approach is applied (Section~\ref{sec-compute-soil-flux}).
Ribbons for modeled soil fluxes represent \(\pm\) 1 standard deviation.
Results are reported in local time.}

\end{figure}%

\begin{figure}

\centering{

\includegraphics{figures/flux-results-year.png}

}

\caption{\label{fig-flux-results-year}Timeseries of both daily-averaged
field \(F_{S}\) (yellow circles) and daily ensemble averaged soil fluxes
(average of \(F_{000}\), \(F_{101}\), \(F_{011}\), \(F_{110}\),
Section~\ref{sec-compute-soil-flux}) by the \texttt{neonSoilFlux} R
package, separated by the diffusivity model used (green or purple lines,
Millington-Quirk or Marshall, Section~\ref{sec-compute-diffusivity}).}

\end{figure}%

\subsection{Assessment of data gaps}\label{assessment-of-data-gaps}

For a given half-hourly time period, the \texttt{neonSoilFlux} packages
assigns a QA flag for a measurement if more one values across all
measurement depths uses gap-filled data (Section~\ref{sec-gapfilling}).
Panel a of Figure~\ref{fig-gap-filled-stats} reports the proportion of
gap-filled data for all input environmental measurements at each site
during the period when field measurements were made. Soil fluxes are
computed from 4 different types of input measurements (\(T_{S}\),
\(SWC\), \(P\), and CO\(_{2}\)), any of which could have a QA flag in a
half-hourly interval. Panel b of Figure~\ref{fig-gap-filled-stats}
displays at each site the distribution of the number of different
gap-filled measurements used to compute a half-hourly flux. The largest
cause of measurements needing to be gap-filled was missing or flagged
soil moisture data. Calculating fluxes for WOOD and SJER required using
the largest proportion of gap-filled measurements, due to substantially
large fractions of flagged or missing \(SWC\) and \(T_{S}\) data.

\begin{figure}

\centering{

\includegraphics{figures/gap-filled-stats.png}

}

\caption{\label{fig-gap-filled-stats}Panel a) Proportion of input
gap-filled environmental measurements used to generate \(F_{S}\) from
the \texttt{neonSoilFlux} package, by study site. Panel b) distribution
of the usage of gap-filled measurements at each site.}

\end{figure}%

\subsection{Assessing the signal to noise ratio (SNR) and evaluating
estimated
uncertainties}\label{assessing-the-signal-to-noise-ratio-snr-and-evaluating-estimated-uncertainties}

The computed signal to noise ratio (SNR) and the proportion of measured
field fluxes within the modeled uncertainty for a given flux computation
method \(F_{ijk}\) suggest that there was substantial variability in the
agreement between the gradient method and field-measured observations
(Figure~\ref{fig-uncertainty-stats}, Section~\ref{sec-post-process}).
Here, values of SNR greater than unity indicate lower reported
uncertainty, as propagated by quadrature due to a relatively higher
precision of measured input variables (CO\(_{2}\), \(T_{S}\), \(SWC\),
or \(P\)).

The sensitivity to an uncertainty reduction factor (\(\epsilon\), bottom
panels in Figure~\ref{fig-uncertainty-stats}) demonstrates how
concordance between measured and modeled fluxes would be affected if
environmental measurement uncertainty \(\sigma_{ijk}\) were to decrease.
As \(\epsilon\) increases from left to right in each figure, the
possible range of values for each predicted flux value decreases and the
proportion of measured fluxes that fall within that range also
decreases.

\begin{figure}

\centering{

\includegraphics{figures/uncertainty-stats.png}

}

\caption{\label{fig-uncertainty-stats}Top panels: distribution of SNR
values across each of the different sites for modeled effluxes from the
\texttt{neonSoilFlux} package, depending on the diffusivity calculation
used (Millington-Quirk or Marshall,
Section~\ref{sec-compute-diffusivity}). Bottom panels: Proportion of
measured \(F_{S}\) within the modeled range of a flux computation method
\(F_{ijk}\) given an uncertainty reduction factor \(\epsilon\), or
\(| F_{S} - F_{ijk} | < (1-\epsilon) \sigma_{ijk}\).}

\end{figure}%

\subsection{Effects of method choice on diffusivity
estimates}\label{effects-of-method-choice-on-diffusivity-estimates}

We assessed the distribution of \(D_{a}\) (from both the Marshall and
Millington-Quirk methods) at each study site, and also computed a
\emph{post hoc} estimate of \(D_{a}\) using field surface flux
measurements (Section~\ref{sec-compute-diffusivity}). For the
field-estimated measurements we omitted negative values of \(D_{a}\), as
that indicates the CO\(_{2}\) gradient decreases with soil depth
(thereby violating assumptions of the flux-gradient method, which is our
focus here). In four of six field sites, the \emph{post hoc} estimate
fell roughly between the two diffusion estimation methods; however this
was less the case in the two driest sites, SJER and SRER
(Table~\ref{tbl-neon-sites}), where the field estimate of diffusivity
was either lower or higher than both of the other methods
(Figure~\ref{fig-diffusivity-plot}).

\begin{figure}

\centering{

\includegraphics{figures/diffusivity-plot.png}

}

\caption{\label{fig-diffusivity-plot}Distribution of diffusivity
(\(D_{a}\)) at each study site. Values of \(D_{a}\) were provided by the
\texttt{neonSoilFlux} package, computed from the Millington-Quirk or
Marshall models (Section~\ref{sec-compute-diffusivity}). A post-hoc
estimate of diffusivity (labeled as ``Field estimated'') was computed
through the field measured flux (Figure~\ref{fig-flux-results}), divided
by the CO\(_{2}\) gradient from the measurement levels closest to the
soil surface, as reported by NEON. We only used \(F_{S}\) measured by
the LICOR 6800 at all sites to standardize comparisons.}

\end{figure}%

\section{Discussion}\label{sec-discussion}

This study presents a unified data science workflow to efficiently
process automated measurements of belowground soil CO\(_{2}\)
concentrations, soil water content, and soil temperature to infer
estimates of soil surface CO\(_{2}\) effluxes through application of
Fick's Law (Equation \ref{eq:ficks}). Our core goals in this study were:
(1) to generate estimates of soil flux from continuous soil sensor data
at terrestrial NEON sites using the flux-gradient method and then (2) to
compare those estimates to field-measured fluxes based on the closed
chamber approach at six NEON focal sites. We discuss our progress toward
these core goals through (1) an overall evaluation of the flux-gradient
approach (and uncertainty calculation) and (2) site-specific evaluation
of differences in estimated vs measured fluxes.

\subsection{General evaluation of flux-gradient
approach}\label{general-evaluation-of-flux-gradient-approach}

Key assumptions of the flux-gradient approach are that CO\(_{2}\)
concentrations increase throughout the soil profile such that the
highest concentrations are observed in the deepest layers. Additionally,
field flux measurements should correlate with \(F_{000}\) because they
represent surface fluxes. Periods where this gradient condition are not
met generally are connected to processes that occur during soil wetting
events, where more shallow soil layers produce higher concentrations of
CO\(_{2}\) due to microbial respiration pulses following rewetting. This
effect is likely to be largest at sites with rich organic soils
(e.g.~KONZ). Based on this reasoning, in these types of situations we
would \emph{a priori} expect
\(F_{011} \leq F_{101} \leq \leq F_{110} \leq F_{000}\) because the
previous flux estimates rely primarily on CO\(_2\) concentrations at
deeper depths, and could miss high concentrations of CO\(_{2}\) produced
in shallower layers.

When modeling soil respiration, typically a non-linear response function
that also considers soil type is used (Bouma \& Bryla, 2000; Yan et al.,
2016, 2018). For the \texttt{neonSoilFlux} package, our characterization
of soil type is connected to the measurement of bulk density, which was
characterized at each NEON site. This bulk density estimate is based on
replicate samples collected from the site megapit at a subset of soil
horizons, with an estimated uncertainty of \(\pm5\%\) (National
Ecological Observatory Network (NEON), 2024c). Coarse fragment estimates
also have very large uncertainties, but because the volume fraction
tends to be low in surface soils it probably wouldn't contribute much
additional flux uncertainty.

Our results suggest that the most important way to improve reliability
of the flux estimate is to reduce the usage of gap-filled data. The
current approach to gap filling in \texttt{neonSoilFlux} uses monthly
mean data to gap fill---this approach decreases the ability of the
estimate to be responsive to short turn pulses that occur with rapid
weather shifts. Four sites (KONZ, SRER, WREF, and UNDE) had more than
75\% of half-hourly periods with no-gap filled measurements. Two sites
(SJER and WOOD) had more than 75\% of half-hourly intervals with just
one gap-filled measurement. While we did not need to use gap-filled
measurements to compute the flux at WREF, field data collection occurred
following a severe rainstorm, with soils at the beginning of the
sampling week near their water holding capacity. We recommend that
whenever possible, knowledge of local field conditions should influence
analysis decisions in addition to any QA filtering protocols in the
\texttt{neonSoilFlux} package.

We recognize that this gap-filling approach may lead to gap-filled
values that are quite different from the actual values, such as an
underestimate of soil moisture following rain events. Further extensions
of the gap filling method could use more sophisticated gap-filling
routines, similar to what is used for net ecosystem carbon exchange
(Falge et al., 2001; Liu et al., 2023; Mariethoz et al., 2015; Moffat et
al., 2007; Zhang et al., 2023). The current gap-filling routine provides
a consistent approach that can be applied to each data stream, but
further work may explore alternative gap-filling approaches.

\subsection{Evaluation of flux-gradient approach at each
site}\label{evaluation-of-flux-gradient-approach-at-each-site}

Derived results from the \texttt{neonSoilFlux} package have patterns
that are broadly consistent with those directly measured in the field
(Figure~\ref{fig-flux-results} and Figure~\ref{fig-flux-results-year}).
One advantage of the \texttt{neonSoilFlux} package is its ability to
calculate fluxes across different soil depths
(Figure~\ref{fig-model-diagram}), which allows for additional
site-specific customization. We believe the package can provide a useful
baseline estimate of soil fluxes that can always be complemented through
additional field measurements.

The six locations studied provide a range of case studies that suggest
different considerations may apply to different sites when applying the
flux-gradient method. For example, the Santa Rita Experimental Range
(SRER) is a desert site characterized by sandy soil, which also was the
location of the highest field soil temperatures that we observed
(\textbf{?@tbl-licor-summary}). At SRER the flux across the top two
layers (\(F_{110}\)) produced a pattern of soil flux most consistent
with the observed field data. The remaining methods \(F_{101}\),
\(F_{011}\), or \(F_{000}\) are derived from information taken from the
deepest layer, which seems to have been decoupled from the surface
layers both in terms of temperature and CO\(_{2}\) concentration. This
may be a general circumstance where there are large diurnal temperature
extremes that rapidly change during the course of a day and overnight,
leading to lags in the timing of when temperature increases propagate
down to deeper soil layers.

Immediately prior to our visit to Konza Prarie (KONZ), that site that
experienced a significant rain event that led to wet soils that
gradually dried out over the course of our time there. This pulse of
precipitation increased the soil CO\(_{2}\) concentration at the top
layer above the concentrations in lower layers, leading to negative
estimated flux values at the start of the experiment. In this case it
was only when the soil began to return to a baseline level that the
assumptions of the flux-gradient method were again met.

Thus, when considering systematic deployment of this method across a
measurement network, there are a number of independent challenges that
require careful consideration. There are clear tradeoffs between (1)
accuracy of modeled fluxes (defined here as closeness to field-measured
\(F_{S}\) and the uncertainty reduction factor \(\epsilon\)), (2)
precision (defined by the SNR), and (3) the choice of the diffusivity
model (Section~\ref{sec-compute-diffusivity}) or flux computation method
(Section~\ref{sec-compute-soil-flux}) used
(Figure~\ref{fig-uncertainty-stats}). There was no predictable pattern
in SNR for either the flux computation method or diffusivity
calculation, indicating that output uncertainty is driven primarily by
input measurement uncertainty (\(T_{S}\), \(P\), \(SWC\), or
CO\(_{2}\)). Across the different flux computation methods, the
proportion of measured fluxes where
\(| F_{S} - F_{ijk} | < (1-\epsilon) \sigma_{ijk}\) decreased as
\(\epsilon\) increased, except where field \(F_{S}\) was already outside
of the modeled range (i.e.~UNDE and WREF). The method \(F_{110}\) (where
soil flux was computed from the top two soil layers) was the least
sensitive to the uncertainty reduction factor (\(\epsilon\)). This lack
of sensitivity could represent that a surface chamber-based measurement
method (e.g.~with a LI-COR instrument) measures the flux up out of the
surface layer and thus is most closely related to assumptions and
measurements inherent in the \(F_{110}\) method.

Finally, comparing the effects of different diffusivity estimation
methods on the match between modeled and measured fluxes
(Figure~\ref{fig-flux-results-year}) highlights the sensitivity of
\(F_{ijk}\) to diffusivity. The comparison between diffusivity estimates
compared to field estimated diffusivity
(Figure~\ref{fig-diffusivity-plot}) demonstrates that site parameters
can dictate which measure of diffusivity is most likely to be accurate
in a given environmental context. Site-specific differences a largely a
reflection of differences in soil moisture across the sites
(Table~\ref{tbl-neon-sites}), as not all diffusivity estimation methods
incorporate soil moisture equivalently. While we here have compares two
approaches to calculate diffusivity (the Millington-Quirk and Marshall
models), it may be valuable to evaluate other diffusivity models
(e.g.~the Moldrup model Moldrup et al. (1999)) as well. Ultimately the
choice of a particular diffusivity model could be determined based on
knowledge of site-specific evaluations or a set of these models could be
used to generate a model ensemble average as a means to trade precision
for a more general approach.

\subsection{Recommendations for future method
development}\label{recommendations-for-future-method-development}

The \texttt{neonSoilFlux} package provides three different approaches to
estimate soil flux using the gradient method. We believe these
approaches enable the software to be used across a range of
site-specific assumptions (Maier \& Schack-Kirchner, 2014). We note,
however, that this choice can have a determinative approach on the
calculated values. Ensemble averaging approaches (Elshall et al., 2018;
Raftery et al., 2005) may be one way to address this problem if the goal
is to calculate fluxes using the same method at a diverse range of
different sites. Two other ideas would be to apply machine learning
algorithms (e.g.~random trees) to generate a single flux estimate across
diverse sites, or using co-located estimates of net ecosystem carbon
exchange from eddy-flux towers to further constrain results or to assess
soil flux results for plausibility.

These challenges notwithstanding, the method used here and made
available in the \texttt{neonSoilFlux} R package has the potential to
produce nearly continuous estimates of flux across all terrestrial NEON
sites. These estimates are a significant improvement on available
approaches to constrain the portion of ecosystem respiration
attributable to the soil. This, in turn, also aids in our ability to
understand the soil contribution to the net ecosystem flux measured at
these sites using the co-located eddy flux towers.

\section{Conclusions}\label{conclusions}

We have here presented an R package \texttt{neonSoilFlux} for the
estimation of soil CO\(_{2}\) fluxes from continuous buried soil sensor
measurements across terrestrial National Ecological Observatory Network
sites. We compared the predicted fluxes to those measured directly using
a field-based closed chamber approach. We find that the flux gradient
method, while broadly effective at producing estimates of flux
comparable to those measured in the field using a chamber-based
technique, is quite sensitive to a number of issues, including most
prominently: missing data (and thus gap-filling of input measurement
datasets) the selection of soil depths used to best calculate the
gradient (which may vary between sites), and finally the choice of
method used for estimating soil diffusivity. Despite these challenges,
the broad geographic scale and high temporal resolution of the NEON data
make a compelling case for continued efforts to refine this approach to
help us understand how soils across diverse ecosystems are responding to
a changing climate.

\section*{Sources Cited}\label{sources-cited}
\addcontentsline{toc}{section}{Sources Cited}

\phantomsection\label{refs}
\begin{CSLReferences}{1}{0}
\bibitem[\citeproctext]{ref-baldocchiMeasuringFluxesTrace2014}
Baldocchi, D. (2014). Measuring fluxes of trace gases and energy between
ecosystems and the atmosphere - the state and future of the eddy
covariance method. \emph{Global Change Biology}, \emph{20}(12),
3600--3609. \url{https://doi.org/10.1111/gcb.12649}

\bibitem[\citeproctext]{ref-berenbaumReportNSFBIO2015}
Berenbaum, M. R., Carpenter, S. R., Hampton, S. E., Running, S. W., \&
Stanzione, D. C. (2015). \emph{Report from the {NSF BIO Advisory
Committee Subcommittee} on {NEON Scope Impacts}}.

\bibitem[\citeproctext]{ref-bond-lambertyNewTechniquesData2018}
Bond-Lamberty, B. (2018). New {Techniques} and {Data} for
{Understanding} the {Global Soil Respiration Flux}. \emph{Earth's
Future}, \emph{6}(9), 1176--1180.
\url{https://doi.org/10.1029/2018EF000866}

\bibitem[\citeproctext]{ref-bond-lambertyTwentyYearsProgress2024}
Bond-Lamberty, B., Ballantyne, A., Berryman, E., Fluet-Chouinard, E.,
Jian, J., Morris, K. A., Rey, A., \& Vargas, R. (2024). Twenty {Years}
of {Progress}, {Challenges}, and {Opportunities} in {Measuring} and
{Understanding Soil Respiration}. \emph{Journal of Geophysical Research:
Biogeosciences}, \emph{129}(2), e2023JG007637.
\url{https://doi.org/10.1029/2023JG007637}

\bibitem[\citeproctext]{ref-bond-lambertyCOSORECommunityDatabase2020}
Bond-Lamberty, B., Christianson, D. S., Malhotra, A., Pennington, S. C.,
Sihi, D., AghaKouchak, A., Anjileli, H., Altaf Arain, M., Armesto, J.
J., Ashraf, S., Ataka, M., Baldocchi, D., Andrew Black, T., Buchmann,
N., Carbone, M. S., Chang, S.-C., Crill, P., Curtis, P. S., Davidson, E.
A., \ldots{} Zou, J. (2020). {COSORE}: {A} community database for
continuous soil respiration and other soil-atmosphere greenhouse gas
flux data. \emph{Global Change Biology}, \emph{26}(12), 7268--7283.
\url{https://doi.org/10.1111/gcb.15353}

\bibitem[\citeproctext]{ref-bond-lambertyGlobalDatabaseSoil2010}
Bond-Lamberty, B., \& Thomson, A. (2010). A global database of soil
respiration data. \emph{Biogeosciences}, \emph{7}(6), 1915--1926.
\url{https://doi.org/10.5194/bg-7-1915-2010}

\bibitem[\citeproctext]{ref-bond-lambertyGlobalRelationshipHeterotrophic2004}
Bond-Lamberty, B., Wang, C., \& Gower, S. T. (2004). A global
relationship between the heterotrophic and autotrophic components of
soil respiration? \emph{Global Change Biology}, \emph{10}(10),
1756--1766. \url{https://doi.org/10.1111/j.1365-2486.2004.00816.x}

\bibitem[\citeproctext]{ref-boumaAssessmentRootSoil2000}
Bouma, T. J., \& Bryla, D. R. (2000). On the assessment of root and soil
respiration for soils of different textures: Interactions with soil
moisture contents and soil {CO2} concentrations. \emph{Plant and Soil},
\emph{227}(1), 215--221. \url{https://doi.org/10.1023/A:1026502414977}

\bibitem[\citeproctext]{ref-chenDoesGeneralTemperatureDependent2005}
Chen, H., \& Tian, H.-Q. (2005). Does a {General Temperature-Dependent
Q10 Model} of {Soil Respiration Exist} at {Biome} and {Global Scale}?
\emph{Journal of Integrative Plant Biology}, \emph{47}(11), 1288--1302.
\url{https://doi.org/10.1111/j.1744-7909.2005.00211.x}

\bibitem[\citeproctext]{ref-davidsonVariabilityRespirationTerrestrial2006}
Davidson, E. A., Janssens, I. A., \& Luo, Y. (2006). On the variability
of respiration in terrestrial ecosystems: Moving beyond {Q10}.
\emph{Global Change Biology}, \emph{12}, 154--164.
\url{https://doi.org/10.1111/j.1365-2486.2005.01065.x}

\bibitem[\citeproctext]{ref-desaiDriversDecadalCarbon2022}
Desai, A. R., Murphy, B. A., Wiesner, S., Thom, J., Butterworth, B. J.,
Koupaei-Abyazani, N., Muttaqin, A., Paleri, S., Talib, A., Turner, J.,
Mineau, J., Merrelli, A., Stoy, P., \& Davis, K. (2022). Drivers of
{Decadal Carbon Fluxes Across Temperate Ecosystems}. \emph{Journal of
Geophysical Research: Biogeosciences}, \emph{127}(12), e2022JG007014.
\url{https://doi.org/10.1029/2022JG007014}

\bibitem[\citeproctext]{ref-elshallRelativeModelScore2018}
Elshall, A. S., Ye, M., Pei, Y., Zhang, F., Niu, G.-Y., \&
Barron-Gafford, G. A. (2018). Relative model score: A scoring rule for
evaluating ensemble simulations with application to microbial soil
respiration modeling. \emph{Stochastic Environmental Research and Risk
Assessment}, \emph{32}(10), 2809--2819.
\url{https://doi.org/10.1007/s00477-018-1592-3}

\bibitem[\citeproctext]{ref-falgeGapFillingStrategies2001}
Falge, E., Baldocchi, D., Olson, R., Anthoni, P., Aubinet, M.,
Bernhofer, C., Burba, G., Ceulemans, R., Clement, R., Dolman, H.,
Granier, A., Gross, P., Grünwald, T., Hollinger, D., Jensen, N.-O.,
Katul, G., Keronen, P., Kowalski, A., Lai, C. T., \ldots{} Wofsy, S.
(2001). Gap filling strategies for defensible annual sums of net
ecosystem exchange. \emph{Agricultural and Forest Meteorology},
\emph{107}(1), 43--69.
\url{https://doi.org/10.1016/S0168-1923(00)00225-2}

\bibitem[\citeproctext]{ref-farranceUncertaintyMeasurementReview2012}
Farrance, I., \& Frenkel, R. (2012).
\href{https://www.ncbi.nlm.nih.gov/pmc/articles/PMC3387884}{Uncertainty
of {Measurement}: {A Review} of the {Rules} for {Calculating Uncertainty
Components} through {Functional Relationships}}. \emph{The Clinical
Biochemist Reviews}, \emph{33}(2), 49--75.

\bibitem[\citeproctext]{ref-friedlingsteinGlobalCarbonBudget2023}
Friedlingstein, P., O'Sullivan, M., Jones, M. W., Andrew, R. M., Bakker,
D. C. E., Hauck, J., Landschützer, P., Le Quéré, C., Luijkx, I. T.,
Peters, G. P., Peters, W., Pongratz, J., Schwingshackl, C., Sitch, S.,
Canadell, J. G., Ciais, P., Jackson, R. B., Alin, S. R., Anthoni, P.,
\ldots{} Zheng, B. (2023). Global {Carbon Budget} 2023. \emph{Earth
System Science Data}, \emph{15}(12), 5301--5369.
\url{https://doi.org/10.5194/essd-15-5301-2023}

\bibitem[\citeproctext]{ref-hamdiSynthesisAnalysisTemperature2013}
Hamdi, S., Moyano, F., Sall, S., Bernoux, M., \& Chevallier, T. (2013).
Synthesis analysis of the temperature sensitivity of soil respiration
from laboratory studies in relation to incubation methods and soil
conditions. \emph{Soil Biology and Biochemistry}, \emph{58}, 115--126.
\url{https://doi.org/10.1016/j.soilbio.2012.11.012}

\bibitem[\citeproctext]{ref-hirano_long-term_2003}
Hirano, T., Kim, H., \& Tanaka, Y. (2003). Long-term half-hourly
measurement of soil {CO2} concentration and soil respiration in a
temperate deciduous forest. \emph{Journal of Geophysical Research:
Atmospheres}, \emph{108}(D20).
\url{https://doi.org/10.1029/2003JD003766}

\bibitem[\citeproctext]{ref-jacksonEcologySoilCarbon2017}
Jackson, R. B., Lajtha, K., Crow, S. E., Hugelius, G., Kramer, M. G., \&
Piñeiro, G. (2017). The {Ecology} of {Soil Carbon}: {Pools},
{Vulnerabilities}, and {Biotic} and {Abiotic Controls}. \emph{Annual
Review of Ecology, Evolution and Systematics}, \emph{48}(Volume 48,
2017), 419--445.
\url{https://doi.org/10.1146/annurev-ecolsys-112414-054234}

\bibitem[\citeproctext]{ref-jianHistoricallyInconsistentProductivity2022}
Jian, J., Bailey, V., Dorheim, K., Konings, A. G., Hao, D., Shiklomanov,
A. N., Snyder, A., Steele, M., Teramoto, M., Vargas, R., \&
Bond-Lamberty, B. (2022). Historically inconsistent productivity and
respiration fluxes in the global terrestrial carbon cycle. \emph{Nature
Communications}, \emph{13}(1), 1733.
\url{https://doi.org/10.1038/s41467-022-29391-5}

\bibitem[\citeproctext]{ref-jianRestructuredUpdatedGlobal2021}
Jian, J., Vargas, R., Anderson-Teixeira, K., Stell, E., Herrmann, V.,
Horn, M., Kholod, N., Manzon, J., Marchesi, R., Paredes, D., \&
Bond-Lamberty, B. (2021). A restructured and updated global soil
respiration database ({SRDB-V5}). \emph{Earth System Science Data},
\emph{13}(2), 255--267. \url{https://doi.org/10.5194/essd-13-255-2021}

\bibitem[\citeproctext]{ref-jiangGlobalSoilRespiration2024}
Jiang, J., Feng, L., Hu, J., Liu, H., Zhu, C., Chen, B., \& Chen, T.
(2024). Global soil respiration predictions with associated
uncertainties from different spatio-temporal data subsets.
\emph{Ecological Informatics}, \emph{82}, 102777.
\url{https://doi.org/10.1016/j.ecoinf.2024.102777}

\bibitem[\citeproctext]{ref-jobbagyVerticalDistributionSoil2000}
Jobbágy, E. G., \& Jackson, R. B. (2000). The {Vertical Distribution} of
{Soil Organic Carbon} and its {Relation} to {Climate} and {Vegetation}.
\emph{Ecological Applications}, \emph{10}(2), 423--436.
\url{https://doi.org/10.1890/1051-0761(2000)010\%5B0423:TVDOSO\%5D2.0.CO;2}

\bibitem[\citeproctext]{ref-liuRobustGapfillingApproach2023}
Liu, K., Li, X., Wang, S., \& Zhang, H. (2023). A robust gap-filling
approach for {European Space Agency Climate Change Initiative} ({ESA
CCI}) soil moisture integrating satellite observations, model-driven
knowledge, and spatiotemporal machine learning. \emph{Hydrology and
Earth System Sciences}, \emph{27}(2), 577--598.
\url{https://doi.org/10.5194/hess-27-577-2023}

\bibitem[\citeproctext]{ref-luoEcologicalForecastingData2011}
Luo, Y., Ogle, K., Tucker, C., Fei, S., Gao, C., LaDeau, S., Clark, J.
S., \& Schimel, D. S. (2011). Ecological forecasting and data
assimilation in a data-rich era. \emph{Ecological Applications},
\emph{21}(5), 1429--1442. \url{https://doi.org/10.1890/09-1275.1}

\bibitem[\citeproctext]{ref-maierUsingGradientMethod2014}
Maier, M., \& Schack-Kirchner, H. (2014). Using the gradient method to
determine soil gas flux: {A} review. \emph{Agricultural and Forest
Meteorology}, \emph{192--193}, 78--95.
\url{https://doi.org/10.1016/j.agrformet.2014.03.006}

\bibitem[\citeproctext]{ref-mariethozFeaturepreservingInterpolationFiltering2015}
Mariethoz, G., Linde, N., Jougnot, D., \& Rezaee, H. (2015).
Feature-preserving interpolation and filtering of environmental time
series. \emph{Environmental Modelling \& Software}, \emph{72}, 71--76.
\url{https://doi.org/10.1016/j.envsoft.2015.07.001}

\bibitem[\citeproctext]{ref-marshallDiffusionGasesPorous1959}
Marshall, T. J. (1959). The {Diffusion} of {Gases Through Porous Media}.
\emph{Journal of Soil Science}, \emph{10}(1), 79--82.
\url{https://doi.org/10.1111/j.1365-2389.1959.tb00667.x}

\bibitem[\citeproctext]{ref-millingtonDiffusionAggregatedPorous1971}
Millington, R. J., \& Shearer, R. C. (1971). Diffusion in aggregated
porous media. \emph{Soil Science}, \emph{111}(6), 372--378.

\bibitem[\citeproctext]{ref-moffatComprehensiveComparisonGapfilling2007}
Moffat, A. M., Papale, D., Reichstein, M., Hollinger, D. Y., Richardson,
A. D., Barr, A. G., Beckstein, C., Braswell, B. H., Churkina, G., Desai,
A. R., Falge, E., Gove, J. H., Heimann, M., Hui, D., Jarvis, A. J.,
Kattge, J., Noormets, A., \& Stauch, V. J. (2007). Comprehensive
comparison of gap-filling techniques for eddy covariance net carbon
fluxes. \emph{Agricultural and Forest Meteorology}, \emph{147}(3),
209--232. \url{https://doi.org/10.1016/j.agrformet.2007.08.011}

\bibitem[\citeproctext]{ref-moldrupModelingDiffusionReaction1999}
Moldrup, P., Olesen, T., Yamaguchi, T., Schjønning, P., \& Rolston, D.
E. (1999). Modeling diffusion and reaction in soils: 9. {The
Buckingham-Burdine-Campbell} equation for gas diffusivity in undisturbed
soil. \emph{Soil Science}, \emph{164}(2), 75.

\bibitem[\citeproctext]{ref-neonBarometricPressure}
National Ecological Observatory Network (NEON). (2024a).
\emph{Barometric pressure ({DP1}.00004.001)}. National Ecological
Observatory Network (NEON). \url{https://doi.org/10.48443/RT4V-KZ04}

\bibitem[\citeproctext]{ref-neonSoilCO2}
National Ecological Observatory Network (NEON). (2024b). \emph{Soil
{CO2} concentration ({DP1}.00095.001)}. National Ecological Observatory
Network (NEON). \url{https://doi.org/10.48443/E7GR-6G94}

\bibitem[\citeproctext]{ref-neonSoilProperties}
National Ecological Observatory Network (NEON). (2024c). \emph{Soil
physical and chemical properties, {Megapit} ({DP1}.00096.001)}. National
Ecological Observatory Network (NEON).
\url{https://doi.org/10.48443/S6ND-Q840}

\bibitem[\citeproctext]{ref-neonSoilTemp}
National Ecological Observatory Network (NEON). (2024d). \emph{Soil
temperature ({DP1}.00041.001)}. National Ecological Observatory Network
(NEON). \url{https://doi.org/10.48443/Q24X-PW21}

\bibitem[\citeproctext]{ref-neonSoilWater}
National Ecological Observatory Network (NEON). (2024e). \emph{Soil
water content and water salinity ({DP1}.00094.001)}. National Ecological
Observatory Network (NEON). \url{https://doi.org/10.48443/A8VY-Y813}

\bibitem[\citeproctext]{ref-normanComparisonSixMethods1997}
Norman, J. M., Kucharik, C. J., Gower, S. T., Baldocchi, D. D., Crill,
P. M., Rayment, M., Savage, K., \& Striegl, R. G. (1997). A comparison
of six methods for measuring soil-surface carbon dioxide fluxes.
\emph{Journal of Geophysical Research: Atmospheres}, \emph{102}(D24),
28771--28777. \url{https://doi.org/10.1029/97JD01440}

\bibitem[\citeproctext]{ref-phillipsValueSoilRespiration2017}
Phillips, C. L., Bond-Lamberty, B., Desai, A. R., Lavoie, M., Risk, D.,
Tang, J., Todd-Brown, K., \& Vargas, R. (2017). The value of soil
respiration measurements for interpreting and modeling terrestrial
carbon cycling. \emph{Plant and Soil}, \emph{413}(1), 1--25.
\url{https://doi.org/10.1007/s11104-016-3084-x}

\bibitem[\citeproctext]{ref-rafteryUsingBayesianModel2005}
Raftery, A. E., Gneiting, T., Balabdaoui, F., \& Polakowski, M. (2005).
\emph{Using {Bayesian Model Averaging} to {Calibrate Forecast
Ensembles}}. \url{https://doi.org/10.1175/MWR2906.1}

\bibitem[\citeproctext]{ref-sallamMeasurementGasDiffusion1984}
Sallam, A., Jury, W. A., \& Letey, J. (1984). Measurement of {Gas
Diffusion Coefficient} under {Relatively Low Air}-filled {Porosity}.
\emph{Soil Science Society of America Journal}, \emph{48}(1), 3--6.
\url{https://doi.org/10.2136/sssaj1984.03615995004800010001x}

\bibitem[\citeproctext]{ref-shaoBioticClimaticControls2015}
Shao, J., Zhou, X., Luo, Y., Li, B., Aurela, M., Billesbach, D.,
Blanken, P. D., Bracho, R., Chen, J., Fischer, M., Fu, Y., Gu, L., Han,
S., He, Y., Kolb, T., Li, Y., Nagy, Z., Niu, S., Oechel, W. C., \ldots{}
Zhang, J. (2015). Biotic and climatic controls on interannual
variability in carbon fluxes across terrestrial ecosystems.
\emph{Agricultural and Forest Meteorology}, \emph{205}, 11--22.
\url{https://doi.org/10.1016/j.agrformet.2015.02.007}

\bibitem[\citeproctext]{ref-shaoSoilMicrobialRespiration2013}
Shao, P., Zeng, X., Moore, D. J. P., \& Zeng, X. (2013). Soil microbial
respiration from observations and {Earth System Models}.
\emph{Environmental Research Letters}, \emph{8}(3), 034034.
\url{https://doi.org/10.1088/1748-9326/8/3/034034}

\bibitem[\citeproctext]{ref-sihiComparingModelsMicrobial2016}
Sihi, D., Gerber, S., Inglett, P. W., \& Inglett, K. S. (2016).
Comparing models of microbial--substrate interactions and their response
to warming. \emph{Biogeosciences}, \emph{13}(6), 1733--1752.
\url{https://doi.org/10.5194/bg-13-1733-2016}

\bibitem[\citeproctext]{ref-tangAssessingSoilCO22003}
Tang, J., Baldocchi, D. D., Qi, Y., \& Xu, L. (2003). Assessing soil
{CO2} efflux using continuous measurements of {CO2} profiles in soils
with small solid-state sensors. \emph{Agricultural and Forest
Meteorology}, \emph{118}(3), 207--220.
\url{https://doi.org/10.1016/S0168-1923(03)00112-6}

\bibitem[\citeproctext]{ref-tangContinuousMeasurementsSoil2005}
Tang, J., Misson, L., Gershenson, A., Cheng, W., \& Goldstein, A. H.
(2005). Continuous measurements of soil respiration with and without
roots in a ponderosa pine plantation in the {Sierra Nevada Mountains}.
\emph{Agricultural and Forest Meteorology}, \emph{132}(3), 212--227.
\url{https://doi.org/10.1016/j.agrformet.2005.07.011}

\bibitem[\citeproctext]{ref-taylorIntroductionErrorAnalysis2022}
Taylor, J. R. (2022). \emph{An {Introduction} to {Error Analysis}: {The
Study} of {Uncertainties} in {Physical Measurements}, {Third Edition}}
(3rd ed.). University Science Press.

\bibitem[\citeproctext]{ref-yanMoistureFunctionSoil2018}
Yan, Z., Bond-Lamberty, B., Todd-Brown, K. E., Bailey, V. L., Li, S.,
Liu, C., \& Liu, C. (2018). A moisture function of soil heterotrophic
respiration that incorporates microscale processes. \emph{Nature
Communications}, \emph{9}(1), 2562.
\url{https://doi.org/10.1038/s41467-018-04971-6}

\bibitem[\citeproctext]{ref-yanPorescaleInvestigationResponse2016}
Yan, Z., Liu, C., Todd-Brown, K. E., Liu, Y., Bond-Lamberty, B., \&
Bailey, V. L. (2016). Pore-scale investigation on the response of
heterotrophic respiration to moisture conditions in heterogeneous soils.
\emph{Biogeochemistry}, \emph{131}(1), 121--134.
\url{https://doi.org/10.1007/s10533-016-0270-0}

\bibitem[\citeproctext]{ref-zhangTemporalGapFilling12Hourly2023}
Zhang, R., Kim, S., Kim, H., Fang, B., Sharma, A., \& Lakshmi, V.
(2023). Temporal {Gap-Filling} of 12-{Hourly SMAP Soil Moisture Over}
the {CONUS Using Water Balance Budgeting}. \emph{Water Resources
Research}, \emph{59}(12), e2023WR034457.
\url{https://doi.org/10.1029/2023WR034457}

\end{CSLReferences}




\end{document}
