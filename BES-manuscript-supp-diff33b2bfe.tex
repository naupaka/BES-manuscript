% Options for packages loaded elsewhere
%DIF LATEXDIFF DIFFERENCE FILE
%DIF DEL BES-manuscript-supp-oldtmp-84584.tex   Sun Nov 16 10:27:28 2025
%DIF ADD BES-manuscript-supp.tex                Sun Nov 16 10:27:01 2025
%DIF 2a2
% Options for packages loaded elsewhere %DIF > 
%DIF -------
\PassOptionsToPackage{unicode}{hyperref}
\PassOptionsToPackage{hyphens}{url}
\PassOptionsToPackage{dvipsnames,svgnames,x11names}{xcolor}
%
\documentclass[
  letterpaper,
  DIV=11,
  numbers=noendperiod]{scrartcl}
%DIF 10c11
%DIF < 
%DIF -------
\usepackage{xcolor} %DIF > 
%DIF -------
\usepackage{amsmath,amssymb}
%DIF 12a13
\setcounter{secnumdepth}{5} %DIF > 
%DIF -------
\usepackage{iftex}
\ifPDFTeX
  \usepackage[T1]{fontenc}
  \usepackage[utf8]{inputenc}
  \usepackage{textcomp} % provide euro and other symbols
\else % if luatex or xetex
%DIF 18c20
%DIF <   \usepackage{unicode-math}
%DIF -------
  \usepackage{unicode-math} % this also loads fontspec %DIF > 
%DIF -------
  \defaultfontfeatures{Scale=MatchLowercase}
  \defaultfontfeatures[\rmfamily]{Ligatures=TeX,Scale=1}
\fi
\usepackage{lmodern}
\ifPDFTeX\else
  % xetex/luatex font selection
\fi
% Use upquote if available, for straight quotes in verbatim environments
\IfFileExists{upquote.sty}{\usepackage{upquote}}{}
\IfFileExists{microtype.sty}{% use microtype if available
  \usepackage[]{microtype}
  \UseMicrotypeSet[protrusion]{basicmath} % disable protrusion for tt fonts
}{}
\makeatletter
\@ifundefined{KOMAClassName}{% if non-KOMA class
  \IfFileExists{parskip.sty}{%
    \usepackage{parskip}
  }{% else
    \setlength{\parindent}{0pt}
    \setlength{\parskip}{6pt plus 2pt minus 1pt}}
}{% if KOMA class
  \KOMAoptions{parskip=half}}
\makeatother
%DIF 42-44d44
%DIF < \usepackage{xcolor}
%DIF < \setlength{\emergencystretch}{3em} % prevent overfull lines
%DIF < \setcounter{secnumdepth}{5}
%DIF -------
% Make \paragraph and \subparagraph free-standing
%DIF 46a45
\makeatletter %DIF > 
%DIF -------
\ifx\paragraph\undefined\else
  \let\oldparagraph\paragraph
%DIF 48c48-54
%DIF <   \renewcommand{\paragraph}[1]{\oldparagraph{#1}\mbox{}}
%DIF -------
  \renewcommand{\paragraph}{ %DIF > 
    \@ifstar %DIF > 
      \xxxParagraphStar %DIF > 
      \xxxParagraphNoStar %DIF > 
  } %DIF > 
  \newcommand{\xxxParagraphStar}[1]{\oldparagraph*{#1}\mbox{}} %DIF > 
  \newcommand{\xxxParagraphNoStar}[1]{\oldparagraph{#1}\mbox{}} %DIF > 
%DIF -------
\fi
\ifx\subparagraph\undefined\else
  \let\oldsubparagraph\subparagraph
%DIF 52c58-64
%DIF <   \renewcommand{\subparagraph}[1]{\oldsubparagraph{#1}\mbox{}}
%DIF -------
  \renewcommand{\subparagraph}{ %DIF > 
    \@ifstar %DIF > 
      \xxxSubParagraphStar %DIF > 
      \xxxSubParagraphNoStar %DIF > 
  } %DIF > 
  \newcommand{\xxxSubParagraphStar}[1]{\oldsubparagraph*{#1}\mbox{}} %DIF > 
  \newcommand{\xxxSubParagraphNoStar}[1]{\oldsubparagraph{#1}\mbox{}} %DIF > 
%DIF -------
\fi
%DIF 54a66
\makeatother %DIF > 
%DIF -------


%DIF 56-57c69
%DIF < \providecommand{\tightlist}{%
%DIF <   \setlength{\itemsep}{0pt}\setlength{\parskip}{0pt}}\usepackage{longtable,booktabs,array}
%DIF -------
\usepackage{longtable,booktabs,array} %DIF > 
%DIF -------
\usepackage{calc} % for calculating minipage widths
% Correct order of tables after \paragraph or \subparagraph
\usepackage{etoolbox}
\makeatletter
\patchcmd\longtable{\par}{\if@noskipsec\mbox{}\fi\par}{}{}
\makeatother
% Allow footnotes in longtable head/foot
\IfFileExists{footnotehyper.sty}{\usepackage{footnotehyper}}{\usepackage{footnote}}
\makesavenoteenv{longtable}
\usepackage{graphicx}
\makeatletter
%DIF 69-75c81-90
%DIF < \def\maxwidth{\ifdim\Gin@nat@width>\linewidth\linewidth\else\Gin@nat@width\fi}
%DIF < \def\maxheight{\ifdim\Gin@nat@height>\textheight\textheight\else\Gin@nat@height\fi}
%DIF < \makeatother
%DIF < % Scale images if necessary, so that they will not overflow the page
%DIF < % margins by default, and it is still possible to overwrite the defaults
%DIF < % using explicit options in \includegraphics[width, height, ...]{}
%DIF < \setkeys{Gin}{width=\maxwidth,height=\maxheight,keepaspectratio}
%DIF -------
\newsavebox\pandoc@box %DIF > 
\newcommand*\pandocbounded[1]{% scales image to fit in text height/width %DIF > 
  \sbox\pandoc@box{#1}% %DIF > 
  \Gscale@div\@tempa{\textheight}{\dimexpr\ht\pandoc@box+\dp\pandoc@box\relax}% %DIF > 
  \Gscale@div\@tempb{\linewidth}{\wd\pandoc@box}% %DIF > 
  \ifdim\@tempb\p@<\@tempa\p@\let\@tempa\@tempb\fi% select the smaller of both %DIF > 
  \ifdim\@tempa\p@<\p@\scalebox{\@tempa}{\usebox\pandoc@box}% %DIF > 
  \else\usebox{\pandoc@box}% %DIF > 
  \fi% %DIF > 
} %DIF > 
%DIF -------
% Set default figure placement to htbp
%DIF 77d92
%DIF < \makeatletter
%DIF -------
\def\fps@figure{htbp}
\makeatother

%DIF 81a95-108
 %DIF > 
 %DIF > 
 %DIF > 
 %DIF > 
\setlength{\emergencystretch}{3em} % prevent overfull lines %DIF > 
 %DIF > 
\providecommand{\tightlist}{% %DIF > 
  \setlength{\itemsep}{0pt}\setlength{\parskip}{0pt}} %DIF > 
 %DIF > 
 %DIF > 
 %DIF > 
  %DIF > 
 %DIF > 
 %DIF > 
%DIF -------
\usepackage{lineno,setspace}
\linenumbers
\doublespacing
\KOMAoption{captions}{tableheading}
\renewcommand{\figurename}{Figure}
\renewcommand{\thefigure}{S\arabic{figure}}
\makeatletter
\@ifpackageloaded{caption}{}{\usepackage{caption}}
\AtBeginDocument{%
\ifdefined\contentsname
  \renewcommand*\contentsname{Table of contents}
\else
  \newcommand\contentsname{Table of contents}
\fi
\ifdefined\listfigurename
  \renewcommand*\listfigurename{List of Figures}
\else
  \newcommand\listfigurename{List of Figures}
\fi
\ifdefined\listtablename
  \renewcommand*\listtablename{List of Tables}
\else
  \newcommand\listtablename{List of Tables}
\fi
\ifdefined\figurename
  \renewcommand*\figurename{Figure}
\else
  \newcommand\figurename{Figure}
\fi
\ifdefined\tablename
  \renewcommand*\tablename{Table}
\else
  \newcommand\tablename{Table}
\fi
}
\@ifpackageloaded{float}{}{\usepackage{float}}
\floatstyle{ruled}
\@ifundefined{c@chapter}{\newfloat{codelisting}{h}{lop}}{\newfloat{codelisting}{h}{lop}[chapter]}
\floatname{codelisting}{Listing}
\newcommand*\listoflistings{\listof{codelisting}{List of Listings}}
\makeatother
\makeatletter
\makeatother
\makeatletter
\@ifpackageloaded{caption}{}{\usepackage{caption}}
\@ifpackageloaded{subcaption}{}{\usepackage{subcaption}}
\makeatother
%DIF 128-130d156
%DIF < \ifLuaTeX
%DIF <   \usepackage{selnolig}  % disable illegal ligatures
%DIF < \fi
%DIF -------
\usepackage{bookmark}
%DIF 132d157
%DIF < 
%DIF -------
\IfFileExists{xurl.sty}{\usepackage{xurl}}{} % add URL line breaks if available
%DIF 134c158
%DIF < \urlstyle{same} % disable monospaced font for URLs
%DIF -------
\urlstyle{same} %DIF > 
%DIF -------
\hypersetup{
  pdftitle={Supplemental information for neonSoilFlux: An R Package for Continuous Sensor-Based Estimation of Soil CO2 Fluxes},
  colorlinks=true,
  linkcolor={blue},
  filecolor={Maroon},
  citecolor={Blue},
  urlcolor={Blue},
  pdfcreator={LaTeX via pandoc}}
%DIF 143a167
 %DIF > 
%DIF -------

\title{Supplemental information for \texttt{neonSoilFlux}: An R Package
for Continuous Sensor-Based Estimation of Soil CO\textsubscript{2}
Fluxes}
\author{}
\date{}
%DIF PREAMBLE EXTENSION ADDED BY LATEXDIFF
%DIF UNDERLINE PREAMBLE %DIF PREAMBLE
\RequirePackage[normalem]{ulem} %DIF PREAMBLE
\RequirePackage{color}\definecolor{RED}{rgb}{1,0,0}\definecolor{BLUE}{rgb}{0,0,1} %DIF PREAMBLE
\providecommand{\DIFadd}[1]{{\protect\color{blue}\uwave{#1}}} %DIF PREAMBLE
\providecommand{\DIFdel}[1]{{\protect\color{red}\sout{#1}}} %DIF PREAMBLE
%DIF SAFE PREAMBLE %DIF PREAMBLE
\providecommand{\DIFaddbegin}{} %DIF PREAMBLE
\providecommand{\DIFaddend}{} %DIF PREAMBLE
\providecommand{\DIFdelbegin}{} %DIF PREAMBLE
\providecommand{\DIFdelend}{} %DIF PREAMBLE
\providecommand{\DIFmodbegin}{} %DIF PREAMBLE
\providecommand{\DIFmodend}{} %DIF PREAMBLE
%DIF FLOATSAFE PREAMBLE %DIF PREAMBLE
\providecommand{\DIFaddFL}[1]{\DIFadd{#1}} %DIF PREAMBLE
\providecommand{\DIFdelFL}[1]{\DIFdel{#1}} %DIF PREAMBLE
\providecommand{\DIFaddbeginFL}{} %DIF PREAMBLE
\providecommand{\DIFaddendFL}{} %DIF PREAMBLE
\providecommand{\DIFdelbeginFL}{} %DIF PREAMBLE
\providecommand{\DIFdelendFL}{} %DIF PREAMBLE
\newcommand{\DIFscaledelfig}{0.5}
%DIF HIGHLIGHTGRAPHICS PREAMBLE %DIF PREAMBLE
\RequirePackage{settobox} %DIF PREAMBLE
\RequirePackage{letltxmacro} %DIF PREAMBLE
\newsavebox{\DIFdelgraphicsbox} %DIF PREAMBLE
\newlength{\DIFdelgraphicswidth} %DIF PREAMBLE
\newlength{\DIFdelgraphicsheight} %DIF PREAMBLE
% store original definition of \includegraphics %DIF PREAMBLE
\LetLtxMacro{\DIFOincludegraphics}{\includegraphics} %DIF PREAMBLE
\newcommand{\DIFaddincludegraphics}[2][]{{\color{blue}\fbox{\DIFOincludegraphics[#1]{#2}}}} %DIF PREAMBLE
\newcommand{\DIFdelincludegraphics}[2][]{% %DIF PREAMBLE
\sbox{\DIFdelgraphicsbox}{\DIFOincludegraphics[#1]{#2}}% %DIF PREAMBLE
\settoboxwidth{\DIFdelgraphicswidth}{\DIFdelgraphicsbox} %DIF PREAMBLE
\settoboxtotalheight{\DIFdelgraphicsheight}{\DIFdelgraphicsbox} %DIF PREAMBLE
\scalebox{\DIFscaledelfig}{% %DIF PREAMBLE
\parbox[b]{\DIFdelgraphicswidth}{\usebox{\DIFdelgraphicsbox}\\[-\baselineskip] \rule{\DIFdelgraphicswidth}{0em}}\llap{\resizebox{\DIFdelgraphicswidth}{\DIFdelgraphicsheight}{% %DIF PREAMBLE
\setlength{\unitlength}{\DIFdelgraphicswidth}% %DIF PREAMBLE
\begin{picture}(1,1)% %DIF PREAMBLE
\thicklines\linethickness{2pt} %DIF PREAMBLE
{\color[rgb]{1,0,0}\put(0,0){\framebox(1,1){}}}% %DIF PREAMBLE
{\color[rgb]{1,0,0}\put(0,0){\line( 1,1){1}}}% %DIF PREAMBLE
{\color[rgb]{1,0,0}\put(0,1){\line(1,-1){1}}}% %DIF PREAMBLE
\end{picture}% %DIF PREAMBLE
}\hspace*{3pt}}} %DIF PREAMBLE
} %DIF PREAMBLE
\LetLtxMacro{\DIFOaddbegin}{\DIFaddbegin} %DIF PREAMBLE
\LetLtxMacro{\DIFOaddend}{\DIFaddend} %DIF PREAMBLE
\LetLtxMacro{\DIFOdelbegin}{\DIFdelbegin} %DIF PREAMBLE
\LetLtxMacro{\DIFOdelend}{\DIFdelend} %DIF PREAMBLE
\DeclareRobustCommand{\DIFaddbegin}{\DIFOaddbegin \let\includegraphics\DIFaddincludegraphics} %DIF PREAMBLE
\DeclareRobustCommand{\DIFaddend}{\DIFOaddend \let\includegraphics\DIFOincludegraphics} %DIF PREAMBLE
\DeclareRobustCommand{\DIFdelbegin}{\DIFOdelbegin \let\includegraphics\DIFdelincludegraphics} %DIF PREAMBLE
\DeclareRobustCommand{\DIFdelend}{\DIFOaddend \let\includegraphics\DIFOincludegraphics} %DIF PREAMBLE
\LetLtxMacro{\DIFOaddbeginFL}{\DIFaddbeginFL} %DIF PREAMBLE
\LetLtxMacro{\DIFOaddendFL}{\DIFaddendFL} %DIF PREAMBLE
\LetLtxMacro{\DIFOdelbeginFL}{\DIFdelbeginFL} %DIF PREAMBLE
\LetLtxMacro{\DIFOdelendFL}{\DIFdelendFL} %DIF PREAMBLE
\DeclareRobustCommand{\DIFaddbeginFL}{\DIFOaddbeginFL \let\includegraphics\DIFaddincludegraphics} %DIF PREAMBLE
\DeclareRobustCommand{\DIFaddendFL}{\DIFOaddendFL \let\includegraphics\DIFOincludegraphics} %DIF PREAMBLE
\DeclareRobustCommand{\DIFdelbeginFL}{\DIFOdelbeginFL \let\includegraphics\DIFdelincludegraphics} %DIF PREAMBLE
\DeclareRobustCommand{\DIFdelendFL}{\DIFOaddendFL \let\includegraphics\DIFOincludegraphics} %DIF PREAMBLE
%DIF AMSMATHULEM PREAMBLE %DIF PREAMBLE
\makeatletter %DIF PREAMBLE
\let\sout@orig\sout %DIF PREAMBLE
\renewcommand{\sout}[1]{\ifmmode\text{\sout@orig{\ensuremath{#1}}}\else\sout@orig{#1}\fi} %DIF PREAMBLE
\makeatother %DIF PREAMBLE
%DIF COLORLISTINGS PREAMBLE %DIF PREAMBLE
\RequirePackage{listings} %DIF PREAMBLE
\RequirePackage{color} %DIF PREAMBLE
\lstdefinelanguage{DIFcode}{ %DIF PREAMBLE
%DIF DIFCODE_UNDERLINE %DIF PREAMBLE
  moredelim=[il][\color{red}\sout]{\%DIF\ <\ }, %DIF PREAMBLE
  moredelim=[il][\color{blue}\uwave]{\%DIF\ >\ } %DIF PREAMBLE
} %DIF PREAMBLE
\lstdefinestyle{DIFverbatimstyle}{ %DIF PREAMBLE
	language=DIFcode, %DIF PREAMBLE
	basicstyle=\ttfamily, %DIF PREAMBLE
	columns=fullflexible, %DIF PREAMBLE
	keepspaces=true %DIF PREAMBLE
} %DIF PREAMBLE
\lstnewenvironment{DIFverbatim}{\lstset{style=DIFverbatimstyle}}{} %DIF PREAMBLE
\lstnewenvironment{DIFverbatim*}{\lstset{style=DIFverbatimstyle,showspaces=true}}{} %DIF PREAMBLE
\lstset{extendedchars=\true,inputencoding=utf8}

%DIF END PREAMBLE EXTENSION ADDED BY LATEXDIFF

\begin{document}
\maketitle


\section{Assessment of data gaps}\label{assessment-of-data-gaps}

For a given half-hourly time period, the \texttt{neonSoilFlux} packages
assigns a QA flag for a measurement if more one values across all
measurement depths uses gap-filled data (Section 4.2.1 of the main
text). Panel a of Figure~\ref{fig-gap-filled-stats} reports the
proportion of gap-filled data for all input environmental measurements
at each site during the period when field measurements were made. Soil
fluxes are computed from 4 different types of input measurements
(\(T_{S}\), \(SWC\), \(P\), and CO\(_{2}\)), any of which could have a
QA flag in a half-hourly interval. Panel b of
Figure~\ref{fig-gap-filled-stats} displays at each site the distribution
of the number of different gap-filled measurements used to compute a
half-hourly flux. The largest cause of measurements needing to be
gap-filled was missing or flagged soil moisture data. Calculating fluxes
for WOOD\DIFaddbegin \DIFadd{, WREF, }\DIFaddend and SJER required using the largest proportion of
gap-filled measurements, due to \DIFdelbegin \DIFdel{substantially large fractions of }\DIFdelend flagged or missing \(SWC\) and
\DIFdelbegin \DIFdel{\(T_{S}\) }\DIFdelend \DIFaddbegin \DIFadd{CO\(_{2}\) }\DIFaddend data.

\begin{figure}

\DIFdelbeginFL %DIFDELCMD < \centering{
%DIFDELCMD < 

%DIFDELCMD < \includegraphics{figures/gap-filled-stats.png}
%DIFDELCMD < 

%DIFDELCMD < }
%DIFDELCMD < %%%
\DIFdelendFL \DIFaddbeginFL \centering{

\pandocbounded{\includegraphics[keepaspectratio]{figures/gap-filled-stats.png}}

}
\DIFaddendFL 

\caption{\label{fig-gap-filled-stats}Panel a) Proportion of input
gap-filled environmental measurements used to generate \(F_{S}\) from
the \texttt{neonSoilFlux} package, by study site. Panel b) distribution
of the usage of gap-filled measurements at each site.}

\end{figure}%

\section{Assessing the signal to noise ratio (SNR) and evaluating
estimated
uncertainties}\label{assessing-the-signal-to-noise-ratio-snr-and-evaluating-estimated-uncertainties}

Following collection of field measurements and calculation of the soil
fluxes from \texttt{neonSoilFlux} package, we compared measured
\(F_{S}\) based on closed-dynamic chamber measurements with the LI-COR
instruments to a given soil flux calculation from \texttt{neonSoilFlux}
for each site and flux computation method. Beyond the model statistics
defined in the main text, we computed the signal to noise ratio (SNR),
defined as the ratio of a modeled soil flux (\(F_{ijk}\)) from
\texttt{neonSoilFlux} to its quadrature uncertainty (\(\sigma_{ijk}\)).

We observed that the range of values (\DIFdelbegin \DIFdel{e.g.~}\DIFdelend \DIFaddbegin \emph{\DIFadd{e.g.}}
\DIFaddend \(F_{ijk} \pm \sigma_{ijk}\) was much larger than the measured field
flux. We evaluated \(| F_{S} - F_{ijk} | < (1-\epsilon) \sigma_{ijk}\),
where \(F_{S}\) is a measured field soil flux from the LI-COR 6800 (as
the LI-COR 870/8250 was used at only three sites in 2024 but the 6800
was used at all sites in both years). The parameter \(\epsilon\) was an
uncertainty reduction factor to evaluate how much the quadrature
uncertainty could be reduced while maintaining precision between modeled
\(F_{ijk}\) and measured \(F_{S}\).

The computed signal to noise ratio (SNR) and the proportion of measured
field fluxes within the modeled uncertainty for a given flux computation
method \(F_{ijk}\) suggest that there was substantial variability in the
agreement between the gradient method and field-measured observations
(Figure~\ref{fig-uncertainty-stats}\DIFdelbegin \DIFdel{, Section 4.3 of the main text}\DIFdelend ). Here, values of SNR greater than
unity \DIFaddbegin \DIFadd{(vertical dashed lines in Figure~\ref{fig-uncertainty-stats})
}\DIFaddend indicate lower reported uncertainty, as propagated by quadrature due to
a relatively higher precision of measured input variables (CO\(_{2}\),
\(T_{S}\), \(SWC\), or \(P\)).

The sensitivity to an uncertainty reduction factor (\(\epsilon\), bottom
panels in Figure~\ref{fig-uncertainty-stats}) demonstrates how
concordance between measured and modeled fluxes would be affected if
environmental measurement uncertainty \(\sigma_{ijk}\) were to decrease.
As \(\epsilon\) increases from left to right in each figure, the
possible range of values for each predicted flux value decreases and the
proportion of measured fluxes that fall within that range also
decreases.

\begin{figure}

\DIFdelbeginFL %DIFDELCMD < \centering{
%DIFDELCMD < 

%DIFDELCMD < \includegraphics{figures/uncertainty-stats.png}
%DIFDELCMD < 

%DIFDELCMD < }
%DIFDELCMD < %%%
\DIFdelendFL \DIFaddbeginFL \centering{

\pandocbounded{\includegraphics[keepaspectratio]{figures/uncertainty-stats.png}}

}
\DIFaddendFL 

\caption{\label{fig-uncertainty-stats}Top panels: distribution of SNR
values across each of the different sites for modeled effluxes from the
\texttt{neonSoilFlux} package, depending on the diffusivity calculation
used (Millington-Quirk or Marshall, Section \DIFdelbeginFL \DIFdelFL{4.2}\DIFdelendFL \DIFaddbeginFL \DIFaddFL{3.2}\DIFaddendFL .2 of the main text).
\DIFaddbeginFL \DIFaddFL{Dashed lines indicate a signal to noise ratio of 1. }\DIFaddendFL Bottom panels:
Proportion of measured \(F_{S}\) within the modeled range of a flux
computation method \(F_{ijk}\) given an uncertainty reduction factor
\(\epsilon\), or \(| F_{S} - F_{ijk} | < (1-\epsilon) \sigma_{ijk}\).}

\end{figure}%




\end{document}
